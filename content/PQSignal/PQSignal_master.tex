{

\newcommand{\vs}[1]{\vspace*{-#1mm}}

\subimport{./}{macros.tex}

\chapterquote{Formal Verification of a Post-Quantum Signal Protocol with Tamarin}{TODO}{}{TODO} \label{cha:PQSignal}


\enluminure{T}{he Signal protocol} is used by billions of people for instant 
messaging in applications such as Facebook Messenger, Google Messages, Signal, 
Skype, and WhatsApp. However, advances in quantum computing threaten the 
security of the cornerstone of this protocol: the Diffie-Hellman key exchange. 
There actually are resistant alternatives, called post-quantum secure, but 
replacing the Diffie-Hellman key exchange with these new primitives requires a 
deep revision of the associated security proof. While the security of the 
current Signal protocol has been extensively studied with hand-written proofs 
and computer-verified symbolic analyses, its quantum-resistant variants lack 
symbolic security analyses.

In this work, we present the first symbolic security model for post-quantum 
variants of the Signal protocol. Our model focuses on the core state machines
of the two main sub-protocols of Signal: the X3DH handshake, and the so-called
\emph{\dr} protocol. Then we show, with an automated proof using the Tamarin
prover, that instantiated with the Hashimoto-Katsumata-Kwiatkowski-Prest
post-quantum Signal's handshake from PKC'21, and the Alwen-Coretti-Dodis
KEM-based \dr from EUROCRYPT'19, the resulting post-quantum Signal protocol
has equivalent security properties to its current classical count\-er\-part.

\section{Introduction}\label{PQSignal:sec:introduction}
\subimport{./}{introduction.tex}

\section{A KEM-Based Signal Protocol}
\label{sec:pqsignal}
\subimport{./}{model}

%\subsection{KEM-Signal}
%\input{instantiation.tex}

\section{Tamarin Formal Verification}
\label{sec:verif}
\subimport{./}{security_properties}

\subimport{./}{Annexe.tex}

}
