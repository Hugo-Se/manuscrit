% !TEX root = introduction_master.tex

\section{La cryptographie fondée sur les réseaux euclidiens} % (fold)
\label{sec:reseaux}

%\usepackage{amsmath,amssymb}


%Un \emph{réseau euclidien} $\mathcal{L}$ est un sous-groupe discret de l’espace vectoriel $\mathbb{R}^n$ tel qu’il existe une base $(\mathbf{b}_i)_{i\in[\![1, n]\!]}$ de $\mathbb{R}^n$ telle que
%\[
%\mathcal{L} = \left\{ \sum_{i=1}^n z_i \mathbf{b}_i : z_i \in \mathbb{Z} \right\}.
%\]

\subsection{Définitions}
Un \emph{réseau euclidien} $\mathcal{L}$ est, pour un entier naturel $n$, un sous-groupe discret de l’espace vectoriel $\mathbb{R}^n$ pour lequel il existe une base $(\mathbf{b}_i)_{i\in[\![1, n]\!]}$ de $\mathbb{R}^n$ telle que
\[
\mathcal{L} = \bigoplus^n_{i=1} \mathbb{Z}\mathbf{b}_i\text{.}
\]
%Chaque vecteur dans $\mathcal{L}$ est une combinaison linéaire avec des coefficients entiers des vecteurs de la base. On dit que $\mathcal{L}$ est généré par la base $\{\mathbf{b}_1, \mathbf{b}_2, \ldots, \mathbf{b}_n\}$.


\subsection{Production d'instances difficiles de problèmes sur les réseaux}

Dans un article\cite{STOC:Ajtai96} publié en 1996, Miklós Ajtai présente la réduction de trois problèmes, considérés
difficiles \emph{dans le pire des cas}, au \emph{cas moyen} d’un problème qu’il définit à partir
d’une classe particulière de réseaux.
En d’autres termes, à partir de la difficulté supposée de trois problèmes dans les instances les
plus défavorables à leur résolution --- sans que ces instances spécifiques ne soit précisées ---
Ajtai démontre la difficulté d’un problème dont l’instance correspond à un réseau tiré aléatoirement
dans la classe qu’il spécifie.

Le premier des trois problèmes supposés difficiles, par exemple, consiste à trouver,
approximativement, la plus petite norme parmi celles des vecteurs non nuls d’un réseau.

%La classe \emph{aléatoire} se définit \emph{modulo} un entier $q$ ;
Les réseaux euclidiens de la classe sont définis comme les suites finis de longueur $m$, où $m$ est un entier qui ne dépend que de $n$.

Ajtai démontre que si un algorithme probabilistique qui opère en temps polynomial trouve, avec une
probabilité de $1/2$ au moins, un vecteur court dans un réseau tiré aléatoirement dans cette classe,
il existe un autre algorithme en temps polynomial qui résout les trois problèmes difficiles avec une
probabilité qui tend exponentiellement vers $1$.
En démontrant ainsi la difficulté du problème du plus court vecteur avec des instances aisément
engendrables, Ajtai a ouvert la voie la création de primitives cryptographiques reposant sur des
problèmes attachés aux réseaux euclidiens.



%opère une réduction
%à un facteur polynomial près, 
%autrement dit
%Premièrement, le problème du plus court vecteur non nul dans un réseau
%En 1996, Miklós Ajtai définit une classe de réseaux euclidiens telle 


\vskip1cm
\hrule
\vskip1cm
Le problème du plus proche vecteur (CVP pour \emph{closest vector problem}) est un problème calculatoire fondamental associé aux réseaux euclidiens. Étant donné un réseau $\mathcal{L}$ dans $\mathbb{R}^n$ généré par une base $\{\mathbf{b}_1, \mathbf{b}_2, \ldots, \mathbf{b}_n\}$ et un point quelconque $\mathbf{t}$ dans $\mathbb{R}^n$, le problème CVP demande de trouver le vecteur dans $\mathcal{L}$ qui est le plus proche de $\mathbf{t}$ en termes de distance euclidienne.

Formellement, nous cherchons un vecteur $\mathbf{v}$ dans $\mathcal{L}$ tel que $\|\mathbf{v}-\mathbf{t}\| = \min_{\mathbf{w} \in \mathcal{L}} \|\mathbf{w}-\mathbf{t}\|$.

Le CVP est connu pour être NP-difficile. De plus, il a été démontré que la version d'approximation du CVP, où l'on cherche un vecteur qui est presque le plus proche de $\mathbf{t}$, est également difficile pour certaines approximations. 

\vskip1cm
\hrule
\vskip1cm


Le problème du plus court vecteur (SVP pour \emph{Shortest Vector Problem}) dans un réseau euclidien est un autre problème calculatoire supposé difficile. Étant donné un réseau $\mathcal{L}$ dans $\mathbb{R}^n$ généré par une base $\{\mathbf{b}_1, \mathbf{b}_2, \ldots, \mathbf{b}_n\}$, le SVP demande de trouver le vecteur non nul le plus court dans $\mathcal{L}$. Formellement, nous cherchons un vecteur $\mathbf{v}$ dans $\mathcal{L}$ tel que $\|\mathbf{v}\| = \min_{\mathbf{w} \in \mathcal{L} \setminus \{ \mathbf{0} \}} \|\mathbf{w}\|$. Le SVP est NP-difficile, et même la version d'approximation du SVP est difficile pour certaines approximations.

\vskip1cm
\hrule
\vskip1cm

Le problème de l’\emph{apprentissage avec erreurs}, communément appelé problème LWE (de l’anglais \emph{Learning With Errors}) est un problème calculatoire supposé difficile introduit en 2005 par Oded Regev\cite{STOC:Regev05}, ce qui lui a valu d’obtenir le prix Gödel en 2018.


% avéré être un excellent candidat pour la construction de systèmes cryptographiques résistants aux attaques par ordinateur quantique.


Soit $q$ un entier naturel et $\chi$ une distribution de probabilité sur l'ensemble des entiers. Le problème LWE consiste à distinguer entre deux types de paires $(\mathbf{a}, b)$, où $\mathbf{a}$ est un vecteur à $n$ composantes tiré uniformément au hasard dans $\mathbb{Z}_q^n$ et $b$ est un scalaire dans $\mathbb{Z}_q$. Dans le premier type de paire, $b$ est égal à $\langle\mathbf{a},\mathbf{s}\rangle + e \mod q$ pour un certain vecteur secret $\mathbf{s}$ dans $\mathbb{Z}_q^n$ et un bruit $e$ tiré de la distribution $\chi$. Dans le deuxième type de paire, $b$ est tiré uniformément au hasard dans $\mathbb{Z}_q$.


Les réseaux euclidiens sont un élément clef dans la preuve de la sécurité du problème LWE. En particulier, Regev a montré que si certaines variantes du problème du plus court vecteur (SVP) et du problème du plus proche vecteur (CVP) dans les réseaux euclidiens sont difficiles à résoudre, alors le problème LWE est aussi difficile à résoudre.

Cela a été démontré en montrant une réduction quantique de ces problèmes de réseaux euclidiens vers le problème LWE. En d'autres termes, si nous avions un algorithme efficace pour résoudre le problème LWE, alors nous pourrions utiliser cet algorithme pour résoudre ces problèmes de réseaux euclidiens en temps polynomial, en supposant que nous ayons accès à un ordinateur quantique.


Le problème LWE a trouvé de nombreuses applications en cryptographie, notamment dans la construction de systèmes de chiffrement, de signatures numériques et de protocoles de transfert de clef. La sécurité de ces constructions repose sur la difficulté supposée du problème LWE, qui à son tour est liée à la difficulté de certains problèmes dans les réseaux euclidiens. Cela fait du problème LWE un outil précieux pour la construction de systèmes cryptographiques résistants aux attaques par ordinateur quantique.

% section reseaux (end)
