% !TEX root = introduction_master.tex

\section{Contributions de cette thèse} % (fold)
\label{sec:contributions}


\subsection[Premier article]{Mise en œuvre efficace d’un protocole d’accréditation anonymes post-quantique~\cite{ARES:CRSS23}} % (fold)
\label{sub:paper_1}


Je me suis employé à étudier les systèmes d’accréditations anonymes décrits dans
la littérature et j’ai participé à l’élaboration d’un schéma d’accréditations anonymes post-quantiques optimisé, fondé
sur les problèmes difficiles liés aux réseaux euclidiens, pour ensuite contribuer au développement du premier ensemble de programmes
informatique à sources ouvertes d’accréditations anonymes post-quantiques qui soit une démonstration directe de la
compatibilité d’un tel schéma avec des cas d’usages concrets.
C’est pour le projet européen PROMETHEUS que nous avons développé la permière version de ces programmes, en tant que démonstrateur.
J’ai notamment programmé en langage C certains éléments de bas niveau comme la transformée de Fourier numérique (souvent désignée dans la littérature comme NTT, pour \textit{number theoretic transform}), qui permet la réalisation de multiplications rapides dans les anneaux de polynômes à coefficients entiers modulo un polynôme cyclotomique $X^d + 1$. Pour optimiser ce programme, j’ai mis en œuvre un système de compression reposant principalement sur le système de Colomb-Rice, particulièrement bien adapté à la compression des nombreuses valeurs gaussiennes utilisées dans notre schéma.

% subsection paper_1 (end)

\subsection[Second article]{Vérification formelle d’un protocole Signal post-quantique avec Tamarin~\cite{BCRS}} % (fold)
\label{sub:paper_2}
Le protocole Signal, sur lequel se fondent de nombreux systèmes de messagerie instantanée actuels, étant lui aussi appelé
à connaître une adaptation post-quantique, nous en avons proposé une vérification formelle réalisée avec
Tamarin en mettant l’accent sur ses deux principales composantes, le protocole d’échange de clefs X3DH et le protocole
du « double-cliquet » de gestion des clefs de session.\par
Tamarin est un outil puissant qui permet de modéliser les protocoles cryptographiques, de définir des propriétés de sécurité et de vérifier automatiquement si ces propriétés, exprimées dans la logique du première ordre, sont satisfaites.\par
Cette analyse de sécurité permet de montrer que, instancié avec le protocole d’échange de clefs de Hashimoto-Katsumata-Kwiatkowski-Prest~\cite{PKC:HKKP21} et le double-cliquet de Alwen, Coretti, et Dodis~\cite{EC:AlwCorDod19}, ce protocole Signal post-quantique présente des propriétés de sécurité équivalentes au protocole classique.

% subsection paper_2 (end)

\subsection[Troisième article]{Un protocole pour la vérification sûre de tatouages numériques embarqués dans les modèles d’apprentissage automatique~\cite{IHMMSec:KTBBS21}} % (fold)
\label{sub:paper_3}

\lipsum[34-35]

% subsection paper_3 (end)

% section contributions (end)
