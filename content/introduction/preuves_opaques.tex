% !TEX root = introduction_master.tex

\section{Preuves à divulgation nulle de connaissance} % (fold)
\label{sec:preuves}
Le principe des preuves à divulgation nulle de connaissance, que j’appelle aussi \emph{preuves opaques} repose sur l'interaction entre un
prouveur et un vérificateur.
Le prouveur souhaite convaincre le vérificateur de la vérité d'une assertion sans divulguer
d'informations supplémentaires.
Pour cela, un protocole interactif est établi, où le prouveur envoie des messages au vérificateur et
où ce dernier effectue des vérifications pour s'assurer de la validité de l'assertion.

Le prouveur s'engage sur une certaine valeur secrète, le vérificateur émet un défi basé sur cet
engagement, et le prouveur répond de manière à prouver la validité de l'assertion sans révéler la
valeur secrète.

Les preuves opaques doivent satisfaire trois propriétés essentielles :

\paragraph{La consistance (\emph{completeness} en anglais).}

Le protocole doit être complet, ce qui signifie que si l'assertion est vraie, le vérificateur
acceptera la preuve avec une probabilité élevée.
En d'autres termes, si le prouveur est honnête et suit le protocole correctement, le vérificateur
sera convaincu de la validité de l'assertion.

\paragraph{La robustesse (\emph{soundness}).}

Le protocole doit être sûr, ce qui signifie que si l'assertion est fausse, aucun prouveur ne peut
tromper le vérificateur avec succès, sauf avec une probabilité négligeable.
Le vérificateur doit pouvoir détecter toute tentative de tromperie de la part du prouveur.

\paragraph{L’opacité (\emph{zero-knowledge}).}

Le protocole doit être à divulgation nulle de connaissance, ce qui signifie que la preuve ne doit
révéler aucune information supplémentaire liée à l'assertion autre que le fait qu'elle est
vraie.
Le vérificateur apprend seulement la validité de l'assertion sans obtenir de connaissances sur les
informations secrètes du prouveur.
\newline

Les preuves à divulgation nulle de connaissance non interactives permettent de réduire l'interaction
requise entre le prouveur et le vérificateur, ce qui améliore l'efficacité des protocoles basés sur
les preuves.
La non-interactivité a ouvert la voie à l'utilisation plus généralisée des preuves opaques dans les
systèmes distribués et les environnements où l'interaction en temps réel n'est pas possible.
Ces preuves non interactives sont aussi utilisées par le système d’accréditations anonymes post-quantique sur lequel j’ai travaillé.
Vous trouverez une définition de ce type d’accréditations dans la section suivante.


%Les preuves à divulgation nulle de connaissance ont trouvé des applications dans de nombreux
%domaines de la cryptographie et de la sécurité. Voici quelques exemples d'applications :

%Les preuves opaques sont utilisées pour prouver l'identité ou la possession d'une information sans
%divulguer les détails spécifiques.
%Cela est utile dans les protocoles d'authentification, où les utilisateurs peuvent prouver leur
%identité sans révéler d'informations confidentielles telles que les mots de passe ou les clés secrètes.
%Les protocoles d'identification anonymes permettent également de prouver l'appartenance à un groupe
%spécifique sans révéler l'identité individuelle.

%Les preuves opaques sont utilisées dans les protocoles de confidentialité pour prouver la
%connaissance d'une information confidentielle sans la divulguer réellement.
%Cela est essentiel pour les transactions sécurisées et l'échange de clés cryptographiques, où les
%parties doivent prouver qu'elles possèdent la clé secrète sans la révéler à d'autres.


%Preuves à divulgation nulle de connaissance sans faille (Fail-stop Zero-Knowledge) ?

% section preuves (end)

