% !TEX root = ../content.tex

%\chapterquote{Introduction}{Call me Ishmael}{Moby-Dick}{Herman Melville} % (fold)
\begin{refsegment}
\chapterquote{Introduction}{Je ne pourrais vous dire, les amis, même si je le voulais, si je
suis ou non un espion français, car je suis tenu au plus absolu secret.}{Là où il y a de l’homme...}{\\ Hervé Le Corneur}
\label{cha:introduction}

%\chapterquote{Introduction}{Three may keep a Secret, if two of them are dead.\footnote{Trois personnes peuvent garder un secret, si deux d’entre elles sont mortes.}}{Poor Richard’s Almanack}{Benjamin Franklin} % (fold)

%\enluminure{C}{’est sur des protocoles cryptographiques} que repose la sécurité de nombreux systèmes de notre monde numérique. 


%Des transactions bancaires, aux systèmes de vote électroniques,
%les propriétés

%Ces systèmes sont régi par des protocoles, des ensembles de règles qui régissent le fonctionnement des systèmes cryptographique et l’intéraction des différents agent qui les composent.

%C’est, pour une large part, sur des systèmes cryptographiques que reposent la sécurité.
\enluminure{T}{ant pour ce qui est} de la préservation du secret des informations confidentielles que de l’authentification des entités numériques ou de la vérification de l’intégrité des données, les systèmes cryptographiques apportent une sécurité indispensable à l’existence même du monde numérique tel que nous le connaissons. Ces systèmes cryptographiques sont fondés sur des protocoles, qui sont des ensembles de règles qui déterminent la manière dont les divers agents d’un système doivent fonctionner et interagir.
%Face à la menace que représente
Le possible essor des calculateurs quantiques est de nature à mettre à mal la sécurité de systèmes cryptographiques qui sont parmi les plus utilisés actuellement. Les protocoles se doivent donc d’évoluer en protocoles dits « post-quantiques » pour offrir aux systèmes de demain la meilleure résistance possible face à cette menace.

%sont établis à partir de protocoles, qui sont des ensembles de règles (qui établissent)
%codifie


\onlyif{chap_minitoc}{
\minitoc
}

\clearpage

\subimport{./}{cryptographie.tex}
\subimport{./}{reseaux.tex}
\subimport{./}{preuves_opaques.tex}
\subimport{./}{accréditations_anonymes.tex}
%\subimport{./}{protocole_Signal.tex}
\subimport{./}{contributions.tex}
%\subimport{./}{history.tex}

%% You can list your papers to make sure they appear in the bibliography
%\nocite{EPRINT:Doe36}
%\nocite{CCS:DDMS39}
%\nocite{NDSS:DoeSmi38}

\iftoggle{biblatex}{
\newpage
\sloppy
%% List your contributions
%\printbibliography[heading=selfbib, category=self, title=My Contributions]

\onlyif{chapter_bib}{
\printbibliography[heading=references, segment=\therefsegment]
}
}{
%% List your contributions using a \bibitem list
}
\end{refsegment}
