% !TEX root = introduction_master.tex

\section{Le protocole Signal} % (fold)
\label{sec:signal}

Le protocole Signal est un protocole de chiffrement de bout en bout conçu pour assurer une communication sécurisée et privée. Développé par Open Whisper Systems, il est utilisé dans des applications de messagerie populaires comme Signal, WhatsApp et Facebook Messenger.\par

L'un des principaux sous-protocoles de Signal est X3DH, le « Triple Diffie-Hellman étendu ». X3DH est un protocole d'établissement de clefs qui permet à deux parties, même si elles n'ont pas de clef de session commune, de convenir d’une telle clef de manière sécurisée et privée.\par

X3DH utilise une combinaison de clefs de longue durée et de clefs éphémères pour assurer à la fois la sécurité et le secret des communications. Il implique trois étapes d’échange de clefs Diffie-Hellman, d’où le nom « Triple Diffie-Hellman ». Cela offre la sécurité en avançant, ce qui signifie que si une clef privée est compromise à un moment donné, cela ne compromet pas les sessions passées car chaque session utilise une nouvelle clef éphémère.\par

En plus de X3DH, le protocole Signal utilise un algorithme appelé le algorithme du double cliquet (\emph{double ratchet} en anglais) pour offrir une sécurité encore plus forte et une plus grande confidentialité. Une fois que les deux parties ont établi une clef de session commune en utilisant X3DH, elles utilisent l'algorithme du double cliquet pour échanger des messages.\par

L'algorithme du double cliquet est conçu pour renouveler automatiquement les clefs de chiffrement après chaque message envoyé, en s'assurant que même si une clef est compromise, seuls les messages futurs seront compromis, pas les messages passés.

En somme, le protocole Signal, en combinant X3DH et l'algorithme du double cliquet, offre une sécurité renforcée, en assurant à la fois la confidentialité des communications et une résilience robuste face à la compromission potentielle des clefs.\par


% section signal (end)
