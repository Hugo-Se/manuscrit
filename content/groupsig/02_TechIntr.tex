%\enote{idée}Ajouter que, le fait de travailler avec des nombres premiers $u$ tels que que l'anneau $\R_u$ se split en plus d'éléments permet
%d'accélérer la multiplication de polynomes. Quand dans notre implémentation, la plupart des nombres
%premiers ($\pcomI, \pcomII, Q$) sont choisis de sorte que chaque élément  de $\R_u$ soit réprésenté
%comme 2048 polynomes modulo des polynomes de degre 4 (via la NTT), et que leur multiplicaations est alors la
%multiplication "terme à terme" de ces polys, ce qui est bien plus rapide que si on faisait la multiplication
%directement dans $\R_u$ (ce serait alors une seule multiplication entre polys modulo $X^{8196} + 1$).
%Thomas dit qu'on peut utiliser le terme "partial-splitting", du fait que c'est pas completment split
%(completmeent split, on aurait decompose $X^d+1$ en facteurs de degre 1).

%\enote{idée}Preciser quelque part que nous, on va juste faire Anonymous credential qui dévoile des attributs
%mais pas des propriétés des attributs? (Cependant, les attributs peuvent eux-meêm encoder des
%propriétés : on peut avoir un attribut "mineur", un attribut "majeur".... Histoire d'un peu plus
%préciser le cadre (l'intro 01 était trés générale sur cela).
% CC: disons que c'est une omission

%As explained in our abstract 
%construction~\ref{sec:anonymous_credential_scheme}, It is possible to 
%construct an anonymous credential scheme using a (relaxed) signature 
%scheme and a (relaxed) verifiable encryption scheme that are compatible, 
%a verifiable encryption scheme being simply an encryption scheme that 
%allows to prove some properties about the clear message using the 
%cipher, the properties being encoding\enote{encoded?} using relations : 
%the proof being the proof of the existence of a witness for the element.

We construct an anonymous credential scheme (the generic construction is 
presented in Section~\ref{sec:anonymous_credential_scheme}) using a 
(relaxed) signature scheme and a (relaxed) verifiable encryption scheme 
that are compatible, a verifiable encryption scheme being simply an 
encryption scheme that allows to prove some properties about the 
cleartext using the ciphertext. These properties are encoded using 
relations and we use non-interactive zero-knowledge proofs (NIZK) to 
show the existence of a witness showing that an element fulfills the 
relation (without revealing any other information).
The relaxation aspect essentially means that:
\begin{compactitem}

	\item for the signature: we allow the verification of more 
signatures that the ones that can be
	 constructed by the signature algorithm $\Sign$. These are 
called relaxed signatures, and are in relation with the
	 ``normal'' signatures: each relaxed signature being associated 
to a normal signature,
	 and called its relaxation.
	 For the forgery game, we forbid the adversary to send a 
relaxation of a signature it obtained from the signature oracle.

	\item for the NIZK proofs of the existence of a witness $w$ for
an element $x$, we allow the
		extractor to output a relaxed witness $\bar{w}$ of $x$, 
a relaxed witness of $x$
		being a witness of a relaxation of $x$, for a relation 
associated to the first one,
		called the relaxed relation.

	\item for the verifiable encryption, we allow the decryption 
algorithm applied to a
		ciphertext $\cipher$ and a NIZK proof $\pi$ for this 
cipher to either extract a witness, leading to a cleartext, which is a 
relaxed message, or decrypt the ciphertext leading to a message $\mess$ 
or relaxed message. These two distinct ways have to be compatible in the 
sense that both relaxed messages obtained must lead to the same message.


%	\item for the verifiable encryption, we allow the decryption 
%algorithm applied to a
%		ciphertext $\cipher$ and a NIZK proof $\pi$ for this cipher 
%to lead to a
%		relaxation of a message $\mess$ and the extraction of 
%the relation to give a
%		relaxed witness of $\mess$. 

%\enote{cf rem~\ref{rem:eric_note_02_TechIntr_1}}

\end{compactitem}

The issuer public and secret keys are the public and secret keys of a 
signature scheme and a verifiable encryption scheme.

An improvement upon the protocol in~\cite{CCS:delLyuSei18} is that the 
credential for an identity $\identity$ and attribute $\alpha$ is given 
by a signature of both $\identity$ and $\alpha$, allowing us to use a 
unique group signature for issuing a credential. In our construction,
the signature is a short $(s_{1}, s_{2}, s_{3})$ such that 
\begin{equation}\label{eq:02_TechIntro_sign} \big[\mathbf{a}^T \:\:|\:\: 
\mathbf{b}^T + \identity \mathbf{g}^T \:\:|\:\: (1 \quad a_3)\:\big]
	\begin{bmatrix} s_{1} \\ s_{2} \\ s_{3} \end{bmatrix} =
\hashSIGN(\alpha) \mod  \pcomII
\end{equation}
where $(\mathbf{a},\mathbf{b})$ is the public key and $\mathbf{g} = (1\:\: \delta)$ a gadget matrix.

A presentation of an attribute~$\alpha$ then consists of the encryption of 
$\identity$ and of a zero-knowledge
 proof of knowledge of $\identity$ and a (relaxed) signature of the 
identity and~$\alpha$.

%\ref{rem:eric_note_02_TechIntr_2}.

% deja present plus bas :
 
%In order to be able to use splitting rings (which is the key point 
%allowing us to use NTT and gain in efficiency -- we get 2048 polynomials 
%of degree~4 instead of 2 polynomials of degree~4096), we used the 
%framework of~\cite{C:AttLyuSei20}. It is interesting to note
% that most of the changes implied by this framework don't change the 
%algorithms but only the security proofs of the scheme.

 The verifiable encryption and associated proof need the introduction of 
 a commitment (we choose the one of presented in~\cite{SCN:BDLOP18} and improved in~\cite{C:AttLyuSei20}) which will have two aims.
%defined in (defined in 
%Section~\ref{subsection:constrcom}). This commitment 
First, it is the main tool in order to transform the signature of a 
(visible) identity into a simple linear relation that hides the identity 
and will be used for the NIZK proof (note that the attribute $\alpha$ is 
still visible in the signature). To obtain this relation, the user 
computes two commitments of $\identity$ and $\delta \identity$:
%\ref{rem:eric_note_02_TechIntr_3}. To 
\begin{align*} 
&\mathbf{t} \gets \begin{bmatrix}t_1\\ t_2\end{bmatrix} = \mathbf{A}\mathbf{r} +
\begin{bmatrix}\mathbf{0} \\ \identity \end{bmatrix} 
\!\!\mod \begin{bmatrix}\pcomI \\ \pcomII \end{bmatrix}
\\
&\mathbf{t'} \gets \begin{bmatrix}t'_1\\ t'_2\end{bmatrix} = \mathbf{A}\mathbf{r'} + \begin{bmatrix}\mathbf{0} \\ \delta\identity \end{bmatrix}
\!\!\mod \begin{bmatrix}\pcomI \\ \pcomII \end{bmatrix}
\end{align*}
where $\mathbf{A} = \begin{bmatrix} 1 & a_1 & a_2 \\ 0 & 1 & a_3 \end{bmatrix} \in \R^{3 \times 2}$
and observes that, taking $\bar{\mathbf{r}}$ (resp. $\bar{\mathsf{r}}'$) the last two coordinates of
$\mathbf{r}$ (resp. $\mathbf{r'}$),   $\comsignature = \begin{bmatrix} s_{1} \\ s_{2} \\ s_{3} -
	[\bar{\mathbf{r}}\:\: \bar{\mathbf{r}}']s_{2}
\end{bmatrix}$ is a short solution of : \begin{equation}\label{eq:02_TechIntro_comsign}
\Big[\mathbf{a}^T\:|\: \mathbf{b}^T + \left[ t_2 \:\: t'_2\right] \:|\: \left(1 \:\: a_3\right)\Big] \comsignature = \hashSIGN(\alpha) \mod \pcomII
\end{equation}

Secondly, this $\comsignature$ element is used in two NIZK proofs: The 
first one proves the knowledge of a small solution $\comsignature$, and 
the second one shows (by considering the random coins) that the 
committed value $\identity$ is the same as the encrypted value 
(encrypted using an auxiliary verifiable encryption scheme, introduced 
in\cite{EC:LyuNev17}).

%\ref{rem:eric_note_02_TechIntr_6}).

 Such a verifiable encryption and proof of knowledge was already present in the signature group
 construction of~\cite{CCS:delLyuSei18}. The main differences are as follows :
 \begin{itemize}
	 \item Their signature did not contain the term $\hashSIGN(\alpha)$ :
\begin{align*}
&\big[\mathbf{a}^T \:\:|\:\: \mathbf{b}^T + \identity \mathbf{g}^T \:\:|\:\: (1 \quad a_3)\:\big] \begin{bmatrix} s_{1} \\ s_{2} \\ s_{3} \end{bmatrix} = 0 \mod \pcomII
\end{align*}
This term is present in our scheme, which allows to link the attribute and
the identity, and thus optimize our protocol, by allowing to construct NIZK 
proofs that hide the identity while leaving the attribute visible.

%Preciser que cet $\alpha$ est important pour relier l'attribut � 
%l'identité, d'une façon qui permettra de faire des preuves NIZK qui 
%cachent l'identité mais laisse visible l'attribut.

\item It was necessary that the nonzero differences of challenges were 
invertible modulo multiple
	primes involved in the scheme.
	In order to ensure the existence of enough invertible nonzero 
differences of challenges, 
these primes numbers $u$ were taken in a such a way that $X^d + 1$
	would split into (only) two components in the ring~$\R_u$. We consider 
in this article the framework of~\cite{C:AttLyuSei20} that only
        needs the nonzero differences of challenges to be ``sufficiently 
often'' invertible. This allows to use prime numbers $u$ such that the rings 
$\R_u$ split into (way) more components. This allows for faster polynomial 
multiplications, and thus an efficiency gain. More precisely, when we 
use ``partial splitting'', i.e. when we choose in our implementation 
most prime numbers ($\pcomI, \pcomII, Q$) such that each element of 
$\R_u$ is represented as 2048 polynomials of degree~4 via NTT (instead 
of 2 polynomials of degree~4096 when $X^d+1$ split into two elements), we obtain a term-to-term 
multiplication of these polynomials, which is way faster than a direct 
multiplication in $\R_u$ (which would lead to a unique multiplication 
between polynomials modulo $X^{8196} + 1$).
It is interesting to note
 that most of the changes implied by this framework do not change the 
algorithms but only the security proofs of the scheme.
 \end{itemize}

%\textbf{Rendre un peu plus clair que on devait avoir que 2 composantes 
%pour assurer que y'ait assez d'inversibles}

%\enote{arranger suite phrase} 

As for the anonymity of credentials and identities, the security of our 
scheme relies on the zero-knowledge property of the NIZK proofs, the 
hiding property of the commitment scheme and the IND-CPA security of the 
encryption scheme.  As for the unforgeability and traceability, it 
relies on the special soundness properties of the NIZK proofs and of the 
unforgeability of the underlying signature scheme (formal lemmas and 
proofs can be found in the full version).


%The security of the schemes relies, for the anonymity of credentials and identities, on the
%zero-knowledge of proofs (Lemmas~\ref{lemma:first_part_simulation},~\ref{lemma:second_part_simulation}), hiding of commitment and IND-CPA security of the cipher.

%the security of the schemes relies, for the unforgeability and traceability, of the special soundness
%properties of the NIZK proofs (Lemmas~\ref{lemma:first_part_extraction},
%\ref{lemma:second_part_extraction}) and of the EUF-CMA security of the signature (Proposition~\ref{prop:unforgsign}).

\iffalse
\begin{verbatim}
    _.~"(_.~"(_.~"(_.~"(_.~"(

    Compléter le blabla, c'est sans doute la partie la plus 
importante...

Notes (explications d'Éric):

* insister sur optimisations par rapport à 2018
page 4 id, attribut alpha
commitment éq 1 et 2
ils cachent l'id mais ne la relient pas à l'attribut
voir signature fin de page 3, cest la graine qui relie visiblement 
id et alpha
présentation = relation entre id invisible et alpha visible
pour ça on committe id
avec la relation linéaire de sig, on obtient une nouvelle 
relation linéaire qui cache id
on doit montrer qu'on connaît id, ie prouver id committée, 
s1, s2, s3 connus
-> on fait zk(s1 s2 s3 connus)
on rajoute un enc(id) et on montre que c'est la même id 
que celle qu'on a committée

Page 22 (34 ? en fait 64), il y a les preuves zk
première preuve "l'id committée est connue"
lemme 8 page 24 "on peut ouvrir le commitment sur le message 
chiffré"
deuxième preuve "l'id committée c'est l'id dont je connais 
une sig"
comme le commitment est binding, on sait que c'est le même 
message

schéma
la signature est une graine
puis relation linéaire (signature cachant id mais pas attr)
équation p4
puis preuve...

ppté de la signature = selectively secure puis vraie
(-> adv divisé par nb identités, via un guess)

ppté chiffr vérifiable = algo déchiffrement un peu 
complexe

page 10 subtilité: les deux identities relaxed ne sont pas 
forcément la même, mais tombent sur la même identité
tout court

si on est capable d'extraire le witness, alors le message 
dedans est bien celui qu'on a chiffré 

----

* CC: est-ce qu'il y a un avantage a utiliser 2018/779
+2020/517 plutot que Vadim 2022/284 ? En termes d'implem ?

Eric: dans l'equation p3 on a un H(attribut)
ils ne l'ont pas
eux ont un u fixe
nous c'est mieux car sinon on doit faire plusieurs
signatures de groupe pour un AC
-> nous, on peut mettre en une fois id et attr

on fait 2 signatures : sign(id | attr) et sign(id)
et preuve meme id

c'est une optimisation

Mais on pourrait le faire avec leur schema, donc pas
un avantage en fait...

\end{verbatim}
\fi
