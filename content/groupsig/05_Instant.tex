\subsection{Instanciation}

For lack of space, we do not present the whole instanciation of the 
protocol here. It follows the generic construction presented 
above and we now informally describe the necessary building blocks.

%On ne va pas decrire tout l'algorithme, mais celui-ci suit la construction 
%generale presentee plus haut, et on presente informellement les briques 
%utilisees.

\subsubsubsection{Relaxed Signature} 
\label{sec:instanciation_relaxed_signature_scheme} As explained in 
Section~\ref{sec:technical_details}, the relaxed signature scheme that 
we use is a generalization of the scheme used in~\cite{CCS:delLyuSei18}.

	The algorithms are presented in 
Figure~\ref{fig:signature_algorithms}. They use the same GPV and NTRU 
trapdoors than the original scheme~\cite{CCS:delLyuSei18}, the NTRU 
trapdoor being only used in security proofs.


\begin{figure*}[htp!]
% \begin{figure}[htp!]
\begin{pchstack}[center]
    \procedure[bodylinesep=4pt]{$\SignKeyGen()$}{%
a_3 \sample \R_{\pcomII} \\
	 \mathbf{a} \sample \R^2_{q_{2}}  \\
	    \mathbf{R} =\begin{pcmbox}\begin{pmatrix} r_1 \quad r_2 \\ e_1 \quad e_2 \end{pmatrix}\end{pcmbox} \sample \ballRoneinfty^{2\times 2} \subset \R^{2\times 2}  \\ 
    \mathbf{b} = [b_1 \: b_2 ] \gets \mathbf{a}^T \mathbf{R}  \\
\spk \gets (\mathbf{a}^t, \mathbf{b}^t, a_3)  \in \R^2_{\pcomII} \times \R^2_{\pcomII} \times \R_{\pcomII} \\
\ssk \gets \mathbf{R} \in {\R}^{2\times 2}  \\
	    \pcreturn (\spk, \ssk)}
	    \pchspace
	    \pchspace
    \procedure[bodylinesep=4pt]{$\Sign(\spk, (m, \alpha))$}{%
	    \mathbf{s}_3 \gets \mathcal{D}^2_{\R , \standarddevSignII}  \\ 
	 \mathbf{g}^T = [1 \quad \delta] \pccomment{recall that we note  $\delta = \ceil{\sqrt{\pcomII}}$} \\
    \text{Using GPV trapdoor, samples } \\(\mathbf{s}_1, \mathbf{s}_2) \gets D^2_{\R,
    \standarddevSignI}\times D^2_{\R, \standarddevSignI} \\ 
    \text{ such that} \colon \\
	\hspace{0.2cm}\big[\mathbf{a}^T \:\:|\:\: \mathbf{b}^T + m \mathbf{g}^T \:\:|\:\: (1 \quad a_3)\:\big]
	\begin{pcmbox}\begin{bmatrix} \mathbf{s}_1 \\ \mathbf{s}_2 \\ \mathbf{s}_3
\end{bmatrix}\end{pcmbox}  \t= \hashSIGN(\alpha)}
\end{pchstack}
% \caption{\label{fig:signkeygen_sign}$\SignKeyGen$ and $\Sign$}
% \end{figure}

% \begin{figure}
\begin{pchstack}[center]
	   \procedure[headlinesep=2pt]{$\SignVf\left(\spk, (m, \alpha),
	   \signature\right)$}{%
		   \pccomment{Verification of the signature of a message}\\
		   \pccomment{we note $\signature =(\mathbf{s}_{m,\alpha,1}, \mathbf{s}_{m,\alpha,2}, \mathbf{s}_{m,\alpha,2})$} \\
	 \mathbf{g}^T = [1 \quad \delta] \pccomment{recall that we note  $\delta = \ceil{\sqrt{\pcomII}}$} \\
	    \pcif \colon\\
	    \t\big[\mathbf{a}^T \:\:|\:\: \mathbf{b}^T + m \mathbf{g}^T \:\:|\:\: (1 \quad a_3)\:\big]
	    \begin{pcmbox}\begin{bmatrix} \mathbf{s}_{m,\alpha,1} \\ \mathbf{s}_{m,\alpha,2} \\
	    \mathbf{s}_{m,\alpha,3} \end{bmatrix}\end{pcmbox} = \hashSIGN(\alpha)  \\
 	\t\land\norm{\left(\mathbf{s}_{m, \alpha, 1}, \mathbf{s}_{m,\alpha, 2}\right)} \leq 2\bsignone \\
	\t\land \norm{\mathbf{s}_{m,\alpha, 3}}  \leq 2\bsigntwo  \pcthen\\
	 \t\pcreturn 0 \\
	 \pcelse  \\
	 \t\pcreturn 1
    }
    \pchspace
    \pchspace
    \procedure[headlinesep=2pt]{$\SignVf\left(\spk,((m, \tilde{c}_{\iota}), \alpha),\bar{\signature}\right)$}{%
		   \pccomment{Verification of the signature of a relaxed message}\\
		   \pccomment{we note $\bar{\signature} = {\big(\mathbf{s}_{m,\alpha,1, \iota}, \mathbf{s}_{m,\alpha,2,\iota}, \mathbf{s}_{m,\alpha,3, \iota}\big)}_{\iota \in\II}$} \\
	 \mathbf{g}^T = [1 \quad \delta] \pccomment{recall that we note  $\delta = \ceil{\sqrt{\pcomII}}$} \\
			\pcif \forall \iota \in \II \colon\\
	    \hspace{0.2cm}\big[\mathbf{a}^T \:\:|\:\: \mathbf{b}^T + m \mathbf{g}^T \:\:|\:\: (1 \quad a_3)\:\big]
	    \begin{pcmbox}\begin{bmatrix} \mathbf{s}_{m,\alpha, 1, \iota} \\ \mathbf{s}_{m,\alpha,2, \iota} \\ \mathbf{s}_{m,\alpha, 3, \iota} 
    \end{bmatrix}\end{pcmbox}  = \tilde{c}_{\iota}\hashSIGN(\alpha) \\
	 \t\land \norm{\tilde{c}_\iota}_1 \leq 4\kappa^2     \\
	 \t\land \norm{\left(\mathbf{s}_{m, \alpha, 1, \iota}, \mathbf{s}_{m,\alpha, 2, \iota}\right)} \leq 2\bsignone \\
	 \t\land \norm{\mathbf{s}_{m,\alpha,3, \iota}}  \leq 2\bsigntwo  \\ 
	 \t\land \tilde{c}_\iota \text{ invertible modulo } \phi_\iota \pcthen \\
	 \t\pcreturn 0 \\
	 \pcelse  \\
 \t\pcreturn 1 }
\end{pchstack}
% \caption{\label{fig:signvf}$\SignVf$ for messages and relaxed messages.}
% \end{figure}

\caption{\label{fig:signature_algorithms}Signature algorithms}
\end{figure*}


%\scalebox{0.8}{\vbox{
%% \begin{figure}[htp!]
\begin{pchstack}[center]
    \procedure[bodylinesep=4pt]{$\SignKeyGen()$}{%
a_3 \sample \R_{\pcomII} \\
	 \mathbf{a} \sample \R^2_{q_{2}}  \\
	    \mathbf{R} =\begin{pcmbox}\begin{pmatrix} r_1 \quad r_2 \\ e_1 \quad e_2 \end{pmatrix}\end{pcmbox} \sample \ballRoneinfty^{2\times 2} \subset \R^{2\times 2}  \\ 
    \mathbf{b} = [b_1 \: b_2 ] \gets \mathbf{a}^T \mathbf{R}  \\
\spk \gets (\mathbf{a}^t, \mathbf{b}^t, a_3)  \in \R^2_{\pcomII} \times \R^2_{\pcomII} \times \R_{\pcomII} \\
\ssk \gets \mathbf{R} \in {\R}^{2\times 2}  \\
	    \pcreturn (\spk, \ssk)}
	    \pchspace
	    \pchspace
    \procedure[bodylinesep=4pt]{$\Sign(\spk, (m, \alpha))$}{%
	    \mathbf{s}_3 \gets \mathcal{D}^2_{\R , \standarddevSignII}  \\ 
	 \mathbf{g}^T = [1 \quad \delta] \pccomment{recall that we note  $\delta = \ceil{\sqrt{\pcomII}}$} \\
    \text{Using GPV trapdoor, samples } \\(\mathbf{s}_1, \mathbf{s}_2) \gets D^2_{\R,
    \standarddevSignI}\times D^2_{\R, \standarddevSignI} \\ 
    \text{ such that} \colon \\
	\hspace{0.2cm}\big[\mathbf{a}^T \:\:|\:\: \mathbf{b}^T + m \mathbf{g}^T \:\:|\:\: (1 \quad a_3)\:\big]
	\begin{pcmbox}\begin{bmatrix} \mathbf{s}_1 \\ \mathbf{s}_2 \\ \mathbf{s}_3
\end{bmatrix}\end{pcmbox}  \t= \hashSIGN(\alpha)}
\end{pchstack}
% \caption{\label{fig:signkeygen_sign}$\SignKeyGen$ and $\Sign$}
% \end{figure}

%}
%}
%\vspace{0.1cm}
%\scalebox{0.8}{\vbox{
%% \begin{figure}
\begin{pchstack}[center]
	   \procedure[headlinesep=2pt]{$\SignVf\left(\spk, (m, \alpha),
	   \signature\right)$}{%
		   \pccomment{Verification of the signature of a message}\\
		   \pccomment{we note $\signature =(\mathbf{s}_{m,\alpha,1}, \mathbf{s}_{m,\alpha,2}, \mathbf{s}_{m,\alpha,2})$} \\
	 \mathbf{g}^T = [1 \quad \delta] \pccomment{recall that we note  $\delta = \ceil{\sqrt{\pcomII}}$} \\
	    \pcif \colon\\
	    \t\big[\mathbf{a}^T \:\:|\:\: \mathbf{b}^T + m \mathbf{g}^T \:\:|\:\: (1 \quad a_3)\:\big]
	    \begin{pcmbox}\begin{bmatrix} \mathbf{s}_{m,\alpha,1} \\ \mathbf{s}_{m,\alpha,2} \\
	    \mathbf{s}_{m,\alpha,3} \end{bmatrix}\end{pcmbox} = \hashSIGN(\alpha)  \\
 	\t\land\norm{\left(\mathbf{s}_{m, \alpha, 1}, \mathbf{s}_{m,\alpha, 2}\right)} \leq 2\bsignone \\
	\t\land \norm{\mathbf{s}_{m,\alpha, 3}}  \leq 2\bsigntwo  \pcthen\\
	 \t\pcreturn 0 \\
	 \pcelse  \\
	 \t\pcreturn 1
    }
    \pchspace
    \pchspace
    \procedure[headlinesep=2pt]{$\SignVf\left(\spk,((m, \tilde{c}_{\iota}), \alpha),\bar{\signature}\right)$}{%
		   \pccomment{Verification of the signature of a relaxed message}\\
		   \pccomment{we note $\bar{\signature} = {\big(\mathbf{s}_{m,\alpha,1, \iota}, \mathbf{s}_{m,\alpha,2,\iota}, \mathbf{s}_{m,\alpha,3, \iota}\big)}_{\iota \in\II}$} \\
	 \mathbf{g}^T = [1 \quad \delta] \pccomment{recall that we note  $\delta = \ceil{\sqrt{\pcomII}}$} \\
			\pcif \forall \iota \in \II \colon\\
	    \hspace{0.2cm}\big[\mathbf{a}^T \:\:|\:\: \mathbf{b}^T + m \mathbf{g}^T \:\:|\:\: (1 \quad a_3)\:\big]
	    \begin{pcmbox}\begin{bmatrix} \mathbf{s}_{m,\alpha, 1, \iota} \\ \mathbf{s}_{m,\alpha,2, \iota} \\ \mathbf{s}_{m,\alpha, 3, \iota} 
    \end{bmatrix}\end{pcmbox}  = \tilde{c}_{\iota}\hashSIGN(\alpha) \\
	 \t\land \norm{\tilde{c}_\iota}_1 \leq 4\kappa^2     \\
	 \t\land \norm{\left(\mathbf{s}_{m, \alpha, 1, \iota}, \mathbf{s}_{m,\alpha, 2, \iota}\right)} \leq 2\bsignone \\
	 \t\land \norm{\mathbf{s}_{m,\alpha,3, \iota}}  \leq 2\bsigntwo  \\ 
	 \t\land \tilde{c}_\iota \text{ invertible modulo } \phi_\iota \pcthen \\
	 \t\pcreturn 0 \\
	 \pcelse  \\
 \t\pcreturn 1 }
\end{pchstack}
% \caption{\label{fig:signvf}$\SignVf$ for messages and relaxed messages.}
% \end{figure}

%}
%}

\subsubsubsection{Verifiable Encryption}
\label{sec:instanciation_relaxed_verifiable_encryption}
	Our scheme is based on the verifiable encryption scheme 
presented in~\cite{EC:LyuNev17} and adapted to~\cite{C:AttLyuSei20}. 
The most interesting part is the $\EncRel$ algorithm (presented in 
Figure~\ref{fig:keygen_encryption_algorithm}), due to our new formalism 
for verifiable encryption. As shown in 
Section~\ref{sec:technical_details}, this algorithm builds upon the 
verifiable encryption scheme from~\cite{EC:LyuNev17} and the commitment 
from~\cite{CCS:delLyuSei18}, in order to obtain a short element 
$\comsignature$ satisfying Equation~\eqref{eq:02_TechIntro_comsign}.

Note that according to the way the one-shot verifiable encryption is 
used in the $\EncRel$ algorithm, it does not encrypt the message $m$ but 
the random coin $\textbf{r}$ used to commit 
to~$m$. This is completely equivalent, since one only needs to 
recover~$\textbf{r}$ by using the decryption algorithm and then open the
commitment to get the message~$m$.

%\todoC{corriger notations} 

Furthermore, the decryption algorithm (presented in 
Figure~\ref{fig:decryption_algorithm}) decrypts as in the scheme 
presented in~\cite{EC:LyuNev17} (adapted to the formalism 
of~\cite{C:AttLyuSei20}), then uses the decrypted value to open the commitment
$\mathbf{t}$ and obtain the message.

For a unique attribute, the relations $\relationclear$ et 
$\barrelationclear$ are defined as follows:
	\small
	\begin{align*}
&\relationclear = \Big\{
		\big(\left(\ElemLangrelationclear\right), \signature \big) \\
&\hspace{1cm}\in \SetInstanciationMessSign \times \SetInstanciationAttrAC \times
	\SetInstanciationPublicKeysSign \times \SetInstanciationSignaturesSign \colon \\
&\hspace{1cm}\SignVf(\spk, (m, \alpha), \signature)   = 1 \Big\} \\
&\barrelationclear = \Big\{
	\Big(\big(\ElemLangbarrelationclear\big), \big( ((\bar{m}, {(\tilde{c}_\iota)}_{\iota \in
					\II}), \alpha), {(\signature_\iota)}_{\iota \in \II}\big)\Big) \\
&\hspace{1cm}\in \SetInstanciationRelMessSign \times \SetInstanciationAttrAC \times 
	\SetInstanciationPublicKeysSign \times \SetInstanciationRelSignaturesSign \colon \\ 
&\hspace{1cm}\forall \iota \in \II \colon\SignVf(\spk, (\bar{m}, {(\tilde{c}_\iota)}_{\iota \in
\II}), \alpha),  {(\signature_\iota)}_{\iota \in \II})  = 1 %\\
%&\hspace{1cm}
\land  m = \bar{m}
\Big\} \\
\end{align*}
\normalsize
%= (\mathbf{s}_{m, \alpha, 1},\mathbf{s}_{m, \alpha ,2}, s_{m, \alpha, 3})
 %= (\mathbf{s}_{\bar{m}, \alpha, 1, \iota},\mathbf{s}_{\bar{m}, \alpha ,2, \iota}, s_{\bar{m}, \alpha, 3, \iota}))

We focus on the algorithms $\KeyGen$ and $\EncRel$ since the other algorithms
are based on~\cite{EC:LyuNev17} and adapted to~\cite{C:AttLyuSei20}.

We define $\DomEncRel \subset \SetPublicKeysEnc \times \relationclear$ 
as the subset of public keys of encryption and relation elements that 
share the same element $a_3$.

\begin{figure}[htp!]
\begin{pchstack}[center]
\procedure[headlinesep=2pt, bodylinesep=2pt]{$\EncKeyGen$}{%
a_1, a_2\sample \R_{\pcomI}\\
a_3 \sample \R_{\pcomII} \\
\mathbf{A}_1 \gets \begin{pcmbox}\begin{pmatrix} 1 \quad a_1 \quad a_2\end{pmatrix}\end{pcmbox}\\
\mathbf{A}_2 \gets \begin{pcmbox}\begin{pmatrix} 0 \quad 1 \quad a_3\end{pmatrix}\end{pcmbox}\\
\mathbf{A} \gets \begin{pcmbox}\begin{pmatrix} \mathbf{A}_1 \\ \mathbf{A}_2 \end{pmatrix}\end{pcmbox}\\
a \sample \R_Q\\
\mathbf{s} \sample \ballRoneinfty^3\\
\mathbf{e} \sample \ballRoneinfty^3\\
\mathbf{b} \gets a \mathbf{s} + \mathbf{e} \in \R^3_Q\\
\epk \gets (\mathbf{A}, a,\mathbf{b})\\
\esk \gets \mathbf{s}\\
\pcreturn (\epk,  \esk)
	}
\pchspace[2em]
\procedure[headlinesep=2pt, bodylinesep=2pt]{$\EncRel(\epk, x\in \relationclear)$}{%
\pccomment{we note $x =\left(\left(\ElemLangrelationclear\right), \signature \right)$} \\
\omega \sample \ballRoneinfty \\
r, e_1 \sample \ballRoneinfty \\
\mathbf{e}_2 \sample \ballRoneinfty^3\\
\mathbf{r} = (\mathbf{r}_1, \mathbf{r}_2, \mathbf{r}_3) \sample \ballRoneinfty^3\\
\mathbf{r'} = (\mathbf{r'}_1, \mathbf{r'}_2, \mathbf{r'}_3) \sample \ballRoneinfty^3\\
\encI \gets p(ar + e_1)\\
\encII \gets p(\mathbf{b}\omega + \mathbf{e}_2) + \mathbf{r}\\
\mathbf{t} \gets \mathbf{A}\mathbf{r} + \begin{pcmbox}\begin{bmatrix} 0 \\
m\end{bmatrix}\end{pcmbox} \!\!\mod\begin{pcmbox}\begin{bmatrix} \pcomI \\ \pcomII \end{bmatrix}\end{pcmbox}\\
\mathbf{t'} \gets \mathbf{A}\mathbf{r'} + \begin{pcmbox}\begin{bmatrix} 0 \\ \delta m \end{bmatrix}\end{pcmbox} 
 \!\!\mod\begin{pcmbox}\begin{bmatrix} \pcomI \\ \pcomII \end{bmatrix}\end{pcmbox} \\
\pccomment{we note $\signature = (\mathbf{s}_{m, \alpha, 1},\mathbf{s}_{m, \alpha ,2}, s_{m, \alpha, 3})$} \\
\bar{\mathbf{r}} = (\mathbf{r}_2, \mathbf{r}_3); \bar{\mathbf{r}}' = (\mathbf{r'}_2, \mathbf{r'}_3)\\
	\comsignature \gets \begin{pcmbox}\begin{bmatrix} \mathbf{s}_{m,\alpha,1} \\
		\mathbf{s}_{m,\alpha,2} \\
	\mathbf{s}_{m,\alpha,3} - [\bar{\mathbf{r}}\:\: \bar{\mathbf{r}}'] \mathbf{s}_{m,\alpha,2}\end{bmatrix}\end{pcmbox} \in
	{\R_{q_2}}^{7\times 1}\\
\cipher \gets (\mathbf{t}, \mathbf{t'} , \encI, \encII) \\
\cipherwitness \gets  (m, \alpha, \mathbf{r}, \mathbf{r}',r,e_1,\mathbf{e}_2, \comsignature) \\
\pcreturn ((\epk, \cipher, \spk=(\mathbf{a}^t, \mathbf{b}^t, a_3), \cipherwitness)
}
\end{pchstack}
\caption{\label{fig:keygen_encryption_algorithm}Encryption and encryption of relations algorithms}
\end{figure}

\begin{figure*}[htp!]
	    \procedure[headlinesep=2pt,bodylinesep=2pt]{$\Dec\big(\esk = \mathbf{s},\epk = (\mathbf{A}, a, \mathbf{b}),
	  \cipher = (\mathbf{t}, \mathbf{t}',\encI, \encII)\big)$}{%
\mathbf{tmp\_dec} \gets (\encII - \encI \mathbf{s})
\< \t  \text{set\_index} \gets \{ j \notin J : \overbar{c} \neq 0 \mod \phi_j \}    \\
\pcif \norm{\mathbf{tmp\_dec}}_{\infty} < Q/4\kappa \pcthen \pccomment{ if no $\bar{c}$ is needed}
\<\pccomment{we will use the $\overbar{c}$ for each $i\notin J$ where  it works}\\
\t\pcreturn  (\mathbf{tmp\_dec}, t_2 - \mathbf{A}_2 \:\mathbf{tmp\_dec})
\< \t  \pcfor \iota \in \text{set\_index}\\
	    J \gets \varnothing
\< \t  \t \overbar{c}_\iota \gets \overbar{c}\\
\pcwhile I \neq J \pcdo
\< \t  \t  \mathbf{r}_\iota \gets \overbar{c}\: \mathbf{tmp\_dec}\\
	    \t \iota^* \gets J - I \pccomment{ Can be taken in any prefered order}
\< \t  J \gets J\:\cup \text{set\_index}\\
\t \text{c\_found} \gets \pcfalse
\< \pcendwhile \\
\t  \pcwhile \text{c\_found} \gets \pcfalse
\< \textbf{construct } \mathbf{r} \in \R \text{ such that :}\\
\t  \t  c' \sample \SetChallenges \text{ such that } c \neq c' \mod \phi_{\iota^*}
\< \t \norm{\mathbf{r}}_\infty \leq \pcomI \pcomII Q /2\\
\t  \t  \overbar{c} \gets c - c'
\< \t \forall \iota \in \II\colon\\
\t  \t   \pcif \norm{ \overbar{c}\:\mathbf{tmp\_dec}}_{\infty} \leq Q/4\kappa   \pcthen
\< \t \mathbf{r} \mod \phi_\iota \!\!\mod \pcomI \pcomII Q = \frac{\mathbf{r}_\iota}{\bar{c}_\iota}\mod \phi_\iota \!\!\mod \pcomI \pcomII Q \\
\t  \t   \t  \text{c\_found} \gets \pctrue 
\< m \gets t_2 - \mathbf{A}_2 \mathbf{r}\\
\t \pcendwhile
\< \pcreturn (\mathbf{r}, (m, (c_\iota)_{\iota\in\II}))
	    }
\caption{\label{fig:decryption_algorithm}Decryption algorithm}
\end{figure*}


As explained in Section~\ref{sec:technical_details}, the associated NIZK 
proof then is the union of two NIZK proofs, linked by the commitment:

\begin{compactitem}

	\item a proof of knowledge of a short element
$\comsignature$ satisfying the non-homogeneous linear equation
\eqref{eq:02_TechIntro_comsign},

	\item a proof of knowledge of an element $r$ which is the cleartext
and which opens the commitment on a message $m$.

\end{compactitem}
These two proofs were already used in~\cite{CCS:delLyuSei18} but we adapted 
them to~\cite{C:AttLyuSei20} and also took into account the more general
definition of the signature.

\iffalse
Ainsi, notre algorithme n'effectue pas juste le chiffrement 

Dire que, notre algorithme est basé sur lalgorithme  de chiffrement vérifiable~\cite{EC:LyuNev17} (+
adaptation a 517). La partie la plus interessante est l'algorithme $\EncRel$
(Figure~\ref{fig:keygen_encryption_algorithm}), dû a notre formalisme
différent de la notion de chiffrement vérifiable. 
Comme expliqué dans la section~\ref{sec:technical_details} \enote{j'ai pas 
vrmt checké}, cet
algorithm mélange le chiffrement vérifiable de~\cite{EC:LyuNev17} et le commitment
de~\cite{CCS:delLyuSei18}, afin de pouvoir obtenir un élémént court $\comsignature$
satisfaisant l'équation~\eqref{eq:02_TechIntro_comsign}. 

On peut aussi ajouter que le déchiffrement (Figure~\ref{fig:decryption_algorithm}) effectue le déchiffrement de~\cite{EC:LyuNev17}
(version adapté au formalisme 517) puis utilise la valeur déchiffrée pour ouvrir le commitement $\mathbf{t}$ et obtenir le message.

Note : Le one shot verifiable encryption, tel qu'il est utilise dans l algo EncRel, 
il ne chiffre pas le message m, mais la random coin (r gras) utilisee pour 
commit m. Toutefois, cela revient au meme, car il suffit dans l algo de 
decryption associe, de retrouver r en dechiffrant, puis d ouvrir le commitment 
pour obtenir le message m.


Les relations $\relationclear$ et $\barrelationclear$ (pour un seul attribut) sont, :
	\small
	\begin{align*}
&\relationclear = \Big\{
		\big(\left(\ElemLangrelationclear\right), \signature \big) \\
&\hspace{1cm}\in \SetInstanciationMessSign \times \SetInstanciationAttrAC \times
	\SetInstanciationPublicKeysSign \times \SetInstanciationSignaturesSign \colon \\
&\hspace{1cm}\SignVf(\spk, (m, \alpha), \signature)   = 1 \Big\} \\
&\barrelationclear = \Big\{
	\Big(\big(\ElemLangbarrelationclear\big), \big( ((\bar{m}, {(\tilde{c}_\iota)}_{\iota \in
					\II}), \alpha), {(\signature_\iota)}_{\iota \in \II}\big)\Big) \\
&\hspace{1cm}\in \SetInstanciationRelMessSign \times \SetInstanciationAttrAC \times 
	\SetInstanciationPublicKeysSign \times \SetInstanciationRelSignaturesSign \colon \\ 
&\hspace{1cm}\forall \iota \in \II \colon\SignVf(\spk, (\bar{m}, {(\tilde{c}_\iota)}_{\iota \in
\II}), \alpha),  {(\signature_\iota)}_{\iota \in \II})  = 1 %\\
%&\hspace{1cm}
\land  m = \bar{m}
\Big\} \\
\end{align*}
\normalsize
%= (\mathbf{s}_{m, \alpha, 1},\mathbf{s}_{m, \alpha ,2}, s_{m, \alpha, 3})
 %= (\mathbf{s}_{\bar{m}, \alpha, 1, \iota},\mathbf{s}_{\bar{m}, \alpha ,2, \iota}, s_{\bar{m}, \alpha, 3, \iota}))


On montre le keygen et le encrel (les autres algorithmes sont des adaptations de 
\cite{EC:LyuNev17} à 517:

We define $\DomEncRel \subset \SetPublicKeysEnc \times \relationclear$ as the subset of public keys of
encryption and relation element that share the same element $a_3$.

\begin{figure}[htp!]
\begin{pchstack}[center]
\procedure[headlinesep=2pt, bodylinesep=2pt]{$\EncKeyGen$}{%
a_1, a_2\sample \R_{\pcomI}\\
a_3 \sample \R_{\pcomII} \\
\mathbf{A}_1 \gets \begin{pcmbox}\begin{pmatrix} 1 \quad a_1 \quad a_2\end{pmatrix}\end{pcmbox}\\
\mathbf{A}_2 \gets \begin{pcmbox}\begin{pmatrix} 0 \quad 1 \quad a_3\end{pmatrix}\end{pcmbox}\\
\mathbf{A} \gets \begin{pcmbox}\begin{pmatrix} \mathbf{A}_1 \\ \mathbf{A}_2 \end{pmatrix}\end{pcmbox}\\
a \sample \R_Q\\
\mathbf{s} \sample \ballRoneinfty^3\\
\mathbf{e} \sample \ballRoneinfty^3\\
\mathbf{b} \gets a \mathbf{s} + \mathbf{e} \in \R^3_Q\\
\epk \gets (\mathbf{A}, a,\mathbf{b})\\
\esk \gets \mathbf{s}\\
\pcreturn (\epk,  \esk)
	}
\pchspace[2em]
\procedure[headlinesep=2pt, bodylinesep=2pt]{$\EncRel(\epk, x\in \relationclear)$}{%
\pccomment{we note $x =\left(\left(\ElemLangrelationclear\right), \signature \right)$} \\
\omega \sample \ballRoneinfty \\
r, e_1 \sample \ballRoneinfty \\
\mathbf{e}_2 \sample \ballRoneinfty^3\\
\mathbf{r} = (\mathbf{r}_1, \mathbf{r}_2, \mathbf{r}_3) \sample \ballRoneinfty^3\\
\mathbf{r'} = (\mathbf{r'}_1, \mathbf{r'}_2, \mathbf{r'}_3) \sample \ballRoneinfty^3\\
\encI \gets p(ar + e_1)\\
\encII \gets p(\mathbf{b}\omega + \mathbf{e}_2) + \mathbf{r}\\
\mathbf{t} \gets \mathbf{A}\mathbf{r} + \begin{pcmbox}\begin{bmatrix} 0 \\
m\end{bmatrix}\end{pcmbox} \!\!\mod\begin{pcmbox}\begin{bmatrix} \pcomI \\ \pcomII \end{bmatrix}\end{pcmbox}\\
\mathbf{t'} \gets \mathbf{A}\mathbf{r'} + \begin{pcmbox}\begin{bmatrix} 0 \\ \delta m \end{bmatrix}\end{pcmbox} 
 \!\!\mod\begin{pcmbox}\begin{bmatrix} \pcomI \\ \pcomII \end{bmatrix}\end{pcmbox} \\
\pccomment{we note $\signature = (\mathbf{s}_{m, \alpha, 1},\mathbf{s}_{m, \alpha ,2}, s_{m, \alpha, 3})$} \\
\bar{\mathbf{r}} = (\mathbf{r}_2, \mathbf{r}_3); \bar{\mathbf{r}}' = (\mathbf{r'}_2, \mathbf{r'}_3)\\
	\comsignature \gets \begin{pcmbox}\begin{bmatrix} \mathbf{s}_{m,\alpha,1} \\
		\mathbf{s}_{m,\alpha,2} \\
	\mathbf{s}_{m,\alpha,3} - [\bar{\mathbf{r}}\:\: \bar{\mathbf{r}}'] \mathbf{s}_{m,\alpha,2}\end{bmatrix}\end{pcmbox} \in
	{\R_{q_2}}^{7\times 1}\\
\cipher \gets (\mathbf{t}, \mathbf{t'} , \encI, \encII) \\
\cipherwitness \gets  (m, \alpha, \mathbf{r}, \mathbf{r}',r,e_1,\mathbf{e}_2, \comsignature) \\
\pcreturn ((\epk, \cipher, \spk=(\mathbf{a}^t, \mathbf{b}^t, a_3), \cipherwitness)
}
\end{pchstack}
\caption{\label{fig:keygen_encryption_algorithm}Encryption and encryption of relations algorithms}
\end{figure}

\begin{figure*}[htp!]
	    \procedure[headlinesep=2pt,bodylinesep=2pt]{$\Dec\big(\esk = \mathbf{s},\epk = (\mathbf{A}, a, \mathbf{b}),
	  \cipher = (\mathbf{t}, \mathbf{t}',\encI, \encII)\big)$}{%
\mathbf{tmp\_dec} \gets (\encII - \encI \mathbf{s})
\< \t  \text{set\_index} \gets \{ j \notin J : \overbar{c} \neq 0 \mod \phi_j \}    \\
\pcif \norm{\mathbf{tmp\_dec}}_{\infty} < Q/4\kappa \pcthen \pccomment{ if no $\bar{c}$ is needed}
\<\pccomment{we will use the $\overbar{c}$ for each $i\notin J$ where  it works}\\
\t\pcreturn  (\mathbf{tmp\_dec}, t_2 - \mathbf{A}_2 \:\mathbf{tmp\_dec})
\< \t  \pcfor \iota \in \text{set\_index}\\
	    J \gets \varnothing
\< \t  \t \overbar{c}_\iota \gets \overbar{c}\\
\pcwhile I \neq J \pcdo
\< \t  \t  \mathbf{r}_\iota \gets \overbar{c}\: \mathbf{tmp\_dec}\\
	    \t \iota^* \gets J - I \pccomment{ Can be taken in any prefered order}
\< \t  J \gets J\:\cup \text{set\_index}\\
\t \text{c\_found} \gets \pcfalse
\< \pcendwhile \\
\t  \pcwhile \text{c\_found} \gets \pcfalse
\< \textbf{construct } \mathbf{r} \in \R \text{ such that :}\\
\t  \t  c' \sample \SetChallenges \text{ such that } c \neq c' \mod \phi_{\iota^*}
\< \t \norm{\mathbf{r}}_\infty \leq \pcomI \pcomII Q /2\\
\t  \t  \overbar{c} \gets c - c'
\< \t \forall \iota \in \II\colon\\
\t  \t   \pcif \norm{ \overbar{c}\:\mathbf{tmp\_dec}}_{\infty} \leq Q/4\kappa   \pcthen
\< \t \mathbf{r} \mod \phi_\iota \!\!\mod \pcomI \pcomII Q = \frac{\mathbf{r}_\iota}{\bar{c}_\iota}\mod \phi_\iota \!\!\mod \pcomI \pcomII Q \\
\t  \t   \t  \text{c\_found} \gets \pctrue 
\< m \gets t_2 - \mathbf{A}_2 \mathbf{r}\\
\t \pcendwhile
\< \pcreturn (\mathbf{r}, (m, (c_\iota)_{\iota\in\II}))
	    }
\caption{\label{fig:decryption_algorithm}Decryption algorithm}
\end{figure*}


Comme expliqué dans la section~\ref{sec:technical_details}, la preuve NIZK 
associée est alors une
union de deux preuves NIZK, liées entre elles par le commitment
\begin{itemize}
	\item Une preuve de connaissance d'un élément court $\comsignature$  satisfaisant l'équation
		linéaire non-homogène \eqref{eq:02_TechIntro_comsign}. 
	\item Une preuve de connaissance d'un $r$ qui est le message clair et qui ouvre le
		commitment sur un message $m$
\end{itemize}
Ces deux preuves étaient déjà utilisées dans~\cite{CCS:delLyuSei18} mais nous les avons adaptés pour
517 et nous avons aussi pris en cmopte la définition plus générale de la signature.
\fi

\subsection{Parameters}

%\fullversion{The}{As shown in Section~\ref{sec:security_proofs_ac_scheme}, the} 

The security of the anonymous credential relies on the security of the 
primitives it uses, that are the signature and verifiable encryption scheme, 
defined in 
Section~\ref{sec:instanciation_relaxed_signature_scheme}.

%respectively defined in 
%sections~\ref{sec:instanciation_relaxed_signature_scheme} 
%and~\ref{sec:instanciation_relaxed_verifiable_encryption} .

We list in this section the main conditions needed for these primitives and 
the choices we made. The meaning of the parameters, and the values we 
choose, are indicated in Table~\ref{tab:parameters_ac_scheme}. 

%\fullversion{Details and justifications for these conditions can be 
%found in the full version.}{}

First, a forgery of the signature (with bounds $\bsignone$, $\bsigntwo$) led 
to a solution of $\MSIS_{\pcomII, 1, 4, \bforgsign}$ with
\[  \bforgsign =  8\kappa^2 \sqrt{d}\:\standarddevSignI + \bsignone +
24\sqrt{2}\:d\kappa^2 \standarddevSignI +
    3\sqrt{d}\bsignone + 4 \kappa^2 \sqrt{6d}\:\standarddevSignII + \bsigntwo 
\] 
	The security of the signature scheme thus needs $\MSIS_{\pcomII, 1, 4, 
\bforgsign}$ to be hard, as well as $\NTRU_{\pcomII,\standarddevSignII}$ and 
$\MLWE_{\pcomII,1,\standarddevSignI}$, and a few other conditions.

Then, the security of the verifiable encryption
% The equation~\eqref{eq:condition_verifiable_encryption_com_sign} 
requires that
\begin{equation} \begin{split}
        &4 \kappa \BoundNizkSignOne  \leq \bsignone \\
        &4\kappa \BoundNizkSignOne + 8 \BoundNizkSignTwo \bcom \leq \bsigntwo
    \end{split} \end{equation}

As for the security of the auxiliary commitment scheme used by 
the verifiable encryption, the binding property needs
$\MSIS_{\pcomI, 1,3,8\kappa\bcom}$ to be hard.
The hiding property behaves the same way as in~\cite{CCS:delLyuSei18}, so 
that we need $\MLWE_{\pcomI,X,d}$ and $\MLWE_{\pcomII, X, d}$ to be hard for a 
little number X of samples, with errors sampled in $\ballRoneinfty^d$.
	For the IND-CPA propriety of the underlying encryption scheme of the 
verifiable encryption, we need, as shown in~\cite{EC:LyuNev17}, 
$\MLWE_{Q,1,s}$ hard with a little $s$, and this is already true if $\MLWE$ is 
hard with $\pcomII > Q$.

For the first part of the NIZK proof, using the notations $\bsone = 
2\sqrt{2d}s$ and $\bstwo = \sqrt{6}\left( \sqrt{d} r + 4 d s\right)$, we need
$\MSIS_{\pcomI, 1,3, 4\BoundNizkDecOne}$ to be hard and also the following 
conditions fulfilled:
\begin{equation*}
    \begin{split}
        & \standarddevSignI \geq 11 \kappa \sqrt{17 d \:}  \\
        & \BoundNizkDecOne \geq \sqrt{6 d\:} \standarddevSignI\quad
        \BoundNizkDecTwo \geq \sqrt{10 d\:}\: \standarddevSignI\\
        &\BoundNizkDecOneInfty \geq 12\standarddevSignI\quad
        \BoundNizkDecTwoInfty \geq 12\standarddevSignI  \\
        & 2p \BoundNizkDecTwoInfty (1 + 6d) + p/4\kappa \leq \pencbig /4 
\kappa \\
        & 8 \kappa \BoundNizkDecOneInfty \leq p
    \end{split}
\end{equation*}

For the second part of the NIZK proof, the conditions are as follows:
\begin{align*}
     & \standarddevnizkII \geq 11 \kappa \bsone = 22\kappa \sqrt{2d} \: 
    \standarddevSignI                                                 \\
     & \standarddevnizkIII \geq 11 \kappa \bstwo = 11 \sqrt{6}\: \kappa \left(  
\sqrt{d} \standarddevSignII + 4 d \standarddevSignI\right) \\
     & \BoundNizkSignOne \geq \sqrt{8 d\: }\:\standarddevnizkII \quad 
\BoundNizkSignTwo \geq \sqrt{6 d\: }\: \standarddevnizkIII
\end{align*}

To ensure the correctness of the zero knowledge proofs, we need to consider 
the inversion of the challenges in four rings : $\R_p$, $\R_{\pcomI}$,
$\R_{\pcomII}$, $\R_Q$.

We use~\cite[Lemma~2.1]{CCS:delLyuSei18} with $k=2$ to ensure that 
each $\bar{c} \in \SetDiffChallenges$ is invertible in $\R_p$. We thus
ask $p = 5 \mod 8$.

We also use the framework of~\cite{C:AttLyuSei20} for $Q, \pcomI$ and 
$\pcomII$, with multiple computations of answers by the prover, depending on
the NIZK proofs. This is why we ask the equation~\eqref{eq:primes_rel_517}.

Finally, all these conditions lead to the table of 
parameters presented in~Table~\ref{tab:parameters_ac_scheme}. 
The precise values are chosen using
the lattice estimator~\cite{DBLP:journals/jmc/AlbrechtPS15} to assess
the security of the computational problems based on lattices.

%\enote{traduire}Les valeurs précises étant choisies en utilisant
%the lattice estimator~\cite{DBLP:journals/jmc/AlbrechtPS15}
%pour assurer la sécurité des problèmes de réseaux.

%on explique les conditions qui relient aux pbs de base, puis
%dire qu'on a utilise le LWE/lattice estimator pour estimer/evaluer la 
%difficulte de ces problemes (il est cite en intro 
%\cite{DBLP:journals/jmc/AlbrechtPS15})


\iffalse
\begin{table*}
\begin{center}
%	\begin{longtable}{lll}
	\begin{tabular}{lll}
		\toprule
		Element                                                      & Description                                                                       & Value                                       \\
		\midrule
		d                                                            & such that $\R =
		\ZZ[X]\left(X^d + 1\right)$                                       & 8192                                        \\
		$\theta$                                                     & \begin{tabular}{l} number
			of irreducible divisor of $X^d + 1$ in \\ $\R_u$ for
		$u\in\{Q,p_1,p_2\}$\end{tabular}
									     &  2048                                           \\
		\midrule
		$\tausign$                                                   & 
		\begin{tabular}{l} number of questions asked for each challenges \\ in the second part
			of the NIZK proof\end{tabular}                                   & 1
		\\ $\taudec$ & \begin{tabular}{l}number of questions asked for each challenges \\ in
			the first part of the NIZK proof\end{tabular}
			     &       1                                                                                                                          \\ 
		\midrule
		$\standarddevSignI$                                          & standard deviation
		for the GPV trapdoor                                           & $6\sqrt{d\pcomII}
		\approx  2^{84.5}$      \\
		$\standarddevSignII$                                         & standard deviation
		for the NTRU trapdoor                                          & $2 1.17
		\sqrt{\pcomII} \approx 2^{76.7}$ \\
		\midrule
		$p$                                                          & prime moduli for
		clears in verifiable encryption                                  & $ \approx 2^{75.4}$ (p33, 8.1)              \\
		$Q$                                                          & prime moduli for
		ciphertexts in verifiable encryption                             & $\approx 2^{85.1}$ (p33, 8.1)               \\
		\midrule
		$\pcomI$                                                     & prime moduli for the
		upper part of commitment                                     & $\pcomI \approx
		2^{40.1}$           \\
		$\pcomII$                                                    & prime moduli for the
		lower part of commitment                                     & $\pcomII \approx 2^{150.9}$                    \\
		$\delta$                                                     & $\ceil{\sqrt{\pcomII}}$                                                           & $\pcomII\approx 2^{80}$                     \\
		\midrule
		$\standarddevnizkI$
		                                                             &
		\begin{tabular}{l}standard deviation for discrete gaussian in \\ the NIZK proof \end{tabular}
		                                                             & $11\kappa
									     \sqrt{17}\:d \approx 2^{20}$
		\\ $\standarddevnizkII$
		                                                             & \begin{tabular}{l}standard
			deviation for discrete
			gaussian in \\ the  NIZK proof
		\end{tabular}                                                        & $22
		\kappa \sqrt{2d}\:\standarddevSignI \approx 2^{103.9}$                                                                                                                                                           \\ $\standarddevnizkIII$
		                                                             & \begin{tabular}{l}standard
			deviation for discrete
			gaussian in \\ the NIZK proof
		\end{tabular}                                                        &
		$11\sqrt{6} \kappa \left(\sqrt{d}\standarddevSignII + 4d\standarddevSignI\right)
		\approx 2^{112.2}$                                                                                                              \\ \midrule $\BoundNizkSignOne$   & bound used in NIZK proof              & $88 \kappa d s$                                  \\
		$\BoundNizkSignTwo$                                          & bound used in NIZK
		proof                                                        & $66 \kappa d \left( \standarddevSignII + 4 \sqrt{d} \: \standarddevSignI \right)$                                               \\
		$\bsignone$                                                  & bound used in
		verification of signatures                                      & $88 \kappa d
		\standarddevSignI$                                                                                                                                                                             \\
		$\bsignone$                                                  & bound used in verification of signatures                                          & $11\kappa\sqrt{102}\:d$                     \\
		$\BoundNizkDecTwo$                                           & bound used in NIZK proof                                                          & $11\kappa\sqrt{170}\:d$                     \\
		$\bcom$                                                      &
		\begin{tabular}{l}bound used in verification of commitments \\ and in NIZK proof.  \end{tabular}                                                                                                                                                                     \\
		\midrule
		$\SetChallenges$                                             & space of challenges                                                               & $\ballRoneinfty$                            \\
		$\SetDiffChallenges$                                         & $\{ c - c' : (c,c') \in \SetChallenges, c \neq c'\}$                              &                                             \\
		% $\SetDiffChallenges(\phitau{\tau})$ ($\phitau{\tau} \in \R$) & $\{ \bar{c}\in
		% \SetDiffChallenges\::\: \bar{c} \neq 0 \mod \phitau{\tau}\}$ &                                                                                                                                 \\
		% $\SetDiffChallenges^{\times2}$                               & $\{ \bar{c}\bar{c}'\::\:(\bar{c}, \bar{c}') \in
		% \SetChallenges^2\}$                                          &                                                                                                                                 \\
		% $\SetDiffChallenges(\phitau{\tau})^{\times2}$                & $\{ \bar{c}\bar{c}' :  (\bar{c}, \bar{c}')
		% \in \SetChallenges(\phitau{\tau})^2\}$                       &                                                                                                                                 \\
		$\rho$                                                       & 
		\begin{tabular}{l} probability used to define the distribution of \\ the choice of
		challenges  \end{tabular} & 63/64    \\
		$\distchall{\rho}$,                                            & \begin{tabular}{@{}l@{}} Distribution on $\SetChallenges$ : each coefficient is chosen independently \\ 0 with a probability $\rho$, -1 and 1 with a probability
			$(1-\rho)/2$\end{tabular}                                                        &                                             \\
		$\kappa$                                                     & \begin{tabular}{@{}l@{}}  bound such that the distribution
			$\distchall{\rho}$ output elements  \\
		$c \in \SetChallenges$ such that	$\onenorm{c} \leq \kappa$ with an overwhelming
			probability\end{tabular}                                                        &
		240                                                                                                                                                                                   \\
		\midrule
		$\hashSIGN$                                                  & Hash function for the
		signature.                                                  & $\hashSIGN : \{0,1\}^*
		\rightarrow \R_{q_2}$        \\
		$\hashZK$                                                    & Hash function for the
		NIZK proofs                                                 & $\hashZK : \{0,1\}^*
		\rightarrow \SetChallenges$          \\
	\end{tabular}
%	\end{longtable}
\end{center}
%\end{verbatim}
		\caption{\label{tab:parameters_ac_scheme}Parameters of the Anonymous Credential Scheme}
\end{table*}

\fi
