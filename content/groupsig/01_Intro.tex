\subsubsubsection{Authentication techniques} The traditional way to 
prove one's identity on the Internet is to ask an authority to provide a 
credential. These techniques can deliver a high level of security; 
However, the most commonly used ones introduce some privacy issues. By not 
respecting the user's anonymity, they can make him be traced from server 
to server. For instance, signature schemes suffer from an \emph{over 
identification} issue, meaning that due to the static nature of public 
keys, users have to reveal their identities (i.e., their public keys) to 
be authenticated. \emph{Anonymous credentials}, initially proposed by 
Chaum~\cite{DBLP:journals/cacm/Chaum85}, can be used to solve these 
privacy issues by revealing only the attributes needed for the 
authentication, or proving some properties on them, or answering 
questions relatively to a subset of these attributes, without having to 
reveal more information, such as the identity of the users.

%allow the user to prove some 
%properties on attributes without having to reveal them.

An anonymous credential may be, for example, used to encode identity 
documents issued by the government. Through this credential, the state 
certifies attributes such as citizenship and date of birth. The citizen 
can store this credential, for example, on his or her smartphone and use 
it to prove statements about his or her certified attributes while 
staying unlinkable across services. Nowadays, conventional anonymous 
credential systems are used in more than 500 million trusted platform 
modules embedding direct anonymous attestations and billions of Intel 
processors implementing EPID systems. The IBM Identity Mixer and 
Microsoft U-Prove are also based on conventional anonymous credentials.

In an anonymous credential protocol, a user is collecting credentials 
from various issuers. He or she can finely control which information 
from which credentials he or she presents to which verifier by the means 
of a \emph{presentation token}. 
	An issuer issues credentials to users, thereby vouching for the 
correctness of the attributes contained in the credential with respect 
to the user to whom the credential is issued.
	By verifying the presentation token that a user presents, a 
verifier protects access to a resource or service that it offers by 
imposing restrictions on the credentials that users must own and the 
information from the attributes contained in these credentials that 
users must present in order to access the service. Finally, an inspector 
(or opener) is a trusted authority who can de-anonymize presentation 
tokens under specific circumstances.

%\enote{cf rem~\ref{rem:eric_note_01_Intro_1}}
%\enote{cf rem~\ref{rem:eric_note_01_Intro_2}} 


\subsubsubsection{Constructions of anonymous credentials} 
	Due to their practical interest, these schemes have received a 
lot of attention from the cryptographic community, for 
instance~\cite{10.5555/517876,EC:CamLys01,C:CamLys02,SCN:CamLys02,C:CamLys04,TCC:BCKL08,DBLP:conf/ima/IzabacheneLV11,NDSS:GarGreMie14,AC:CDHK15,ISC:BosCamNev18}.

%\enote{mal ecrit ptet}
To the best of our knowledge, \cite{ISC:BosCamNev18} yields to the only  
post-quantum scheme that has been proposed so far, but with no concrete
instantiation nor implementation.
Indeed, all the others schemes rely on classical assumptions, based for instance on discrete logarithm or Diffie-Hellman. 
It raises concerns about their long-term security, as the emergence of 
quantum computing in recent years threatens any cryptographic protocol 
based on assumptions that are not assumed quantum-safe. Indeed, quantum 
computers would potentially break, even retroactively, the mathematical 
foundations of many current cryptographic systems including the 
difficulty of the Diffie-Hellman problem. 
In response to this potential threat to current cryptographic 
systems, the National Institute of Standards and Technology (NIST) has 
launched a standardization process for post-quantum cryptographic 
primitives in 2016. The goal of this campaign is to provide new 
post-quantum standards for two basic and crucial cryptographic building 
blocks: Key Encapsulation Mechanisms and digital signatures. A few 
families of mathematical problems are considered quantum-safe nowadays 
and can be considered for the design of candidate algorithms like error 
correcting codes or lattices. We focus in this article on lattice-based 
constructions.

% tres inspire de l'intro d'ACNS, je reecrirai si j'ai le temps.

% Therefore, it is crucial to carefully consider the long-term security 
% of anonymous credential protocols and design them accordingly.

	The usual way of constructing an anonymous credential protocol 
is either from a signature scheme with efficient protocols or from group 
signatures. We focus here on the latter case. A group signature scheme, 
first introduced by Chaum and van Heyst in 1991~\cite{EC:ChaVan91}, 
allows a member of a group to anonymously sign a message on behalf of 
the group. It can be used in settings where it is sufficient for a 
verifier to know that a message was sent by a person from a certain 
group, but without needing to know who in particular signed it. Such 
schemes intrinsically allow for anonymity of the users. In our setting, 
it allows to reveal an attribute but not the identity of the user. It 
generally consists of a signature on a committed value, commitments 
being an analogue of a sealed envelope, allowing one to hide a value 
without being able to change his or her mind later on. Intuitively, the 
choice of the signature scheme determines the efficiency and the 
security of the resulting anonymous credential protocol. Each such 
lattice-based signature schemes has its own strengths, but those with 
tight reduction to standard lattice problems are not practical enough to 
lead to practical anonymous credential protocols.

%\enote{cf rem~\ref{rem:eric_note_01_Intro_3}} 
%\enote{cf rem~\ref{rem:eric_note_01_Intro_4}}

% todo


\subsubsubsection{Properties for Anonymous Credentials}
	Three notions capture the security of our anonymous credential 
	protocol: \emph{anonymity}, \emph{unforgeability} and \emph{traceability}. For anonymity, 
we consider a probabilistic polynomial-time adversary who has access to 
all the credentials as well as to a quantum computer with polynomial 
space. The adversary chooses policy rules for the attributes and two 
credentials whose attributes pass the verification of this policy. The 
challenger then returns a presentation token for one of these 
credentials. The adversary's goal is to guess which credential has been 
used to issue this presentation token, i.e. to distinguish between 
presentation tokens for these credentials. 

%The security proof is done by reduction, meaning that the idea is to 
%prove that if an adversary breaks the anonymity of the anonymous 
%credential protocol in quantum polynomial time with a non-negligible 
%advantage, then there is an adversary who breaks the underlying 
%computational problem (such as Ring-LWE or Ring-SIS) in polynomial time 
%with a non-negligible advantage.

% presentation policy -> policy rules
% https://tsapps.nist.gov/publication/get_pdf.cfm?pub_id=919151
%\enote{cf rem~\ref{rem:eric_note_01_Intro_7}} 
Unforgeability captures the notion that an adversary cannot create a valid 
presentation token without using a credential or restraining a valid 
presentation token to a smaller subset of attributes.

Traceability captures the notion that all presentation tokens, even 
when computed by a collusion of users and inspectors, should trace to a 
member of the forging coalition. For the traceability game, the 
adversary has access to all the inspectors' decryption oracles, as well 
as all the credentials for any subset of identities, and has access to a 
quantum computer with polynomial space. The adversary's goal is to 
produce a valid presentation token, i.e. which passes verification, such 
that an inspection of this presentation token fails or traces to an 
identity which is not in the given subset of identities. 


%Full-traceability captures the notion that all presentation tokens, even 
%when computed by a collusion of users and inspectors, should trace to a 
%member of the forging coalition. For the full-traceability game, the 
%adversary has access to all the inspectors' decryption oracles, as well 
%as all the credentials for any subset of identities, and has access to a 
%quantum computer with polynomial space. The adversary's goal is to 
%produce a valid presentation token, i.e. which passes verification, such 
%that an inspection of this presentation token fails or traces to an 
%identity which is not in the given subset of identities. It is worth 
%noting that, since the forgery game is a special case of the 
%full-traceability game, the full-traceability implies both credential 
%unforgeability and presentation token unforgeability.

% secret keys\enote{cf rem~\ref{rem:eric_note_01_Intro_8}}

%Similar to anonymity, the idea is to prove that if an adversary breaks 
%the full-traceability of the anonymous credential protocol in quantum 
%polynomial time with a non-negligible advantage, then there is an 
%adversary who breaks Ring-LWE, Ring-SIS or NTRU in polynomial time with 
%a non-negligible advantage.

For all security properties, the proof is done by reduction, meaning 
that the idea is to prove that if an adversary breaks either property of 
the anonymous credential protocol in quantum polynomial time with a 
non-negligible advantage, then one can construct an adversary which breaks 
the underlying computational problem (such as $\MLWE$, $\MSIS$ or $\NTRU$) in polynomial time with a non-negligible advantage.

\subsubsubsection{Our Contributions} Following the definitions given 
in~\cite{ISC:BosCamNev18} and the framework presented in 
\cite{CCS:delLyuSei18}, we give a generic framework for anonymous 
credential schemes, which encompasses the construction given in 
\cite{ISC:BosCamNev18}.

Furthermore, we use this framework to give a concrete construction of 
lattice-based anonymous credential system, which is instantiated with 
the group signature from \cite{CCS:delLyuSei18} and zero-knowledge 
arguments from \cite{C:AttLyuSei20} (further improved 
in~\cite{C:LyuNguPla22}). We use the lattice 
estimator~\cite{DBLP:journals/jmc/AlbrechtPS15} to estimate the security 
of the resulting anonymous credential protocol and choose a concrete set 
of parameters.

We also give a concrete implementation of our protocol, using the NTT 
along with other optimizations, thanks to the zero-knowledge arguments 
from~\cite{C:AttLyuSei20}. As a result, our implementation of a 
post-quantum anonymous credential is only an order of magnitude slower 
and bandwidth consuming than previous anonymous credentials that are not 
post-quantum. 

%\enote{!}\textbf{NOTE :} ajouter le fait qu'on propose aussi un 
%framework general pour creer des AC schemes, et qu'en plus de notre 
%instanciation, \cite{ISC:BosCamNev18} rentre dans ce framework.

%To build an anonymous credential system it is in addition necessary to 
%append an encryption scheme with zero-knowledge proofs and two 
%compatible zero-knowledge protocols to prove knowledge and equality of 
%two committed values and knowledge of a signature on a committed value. 
%\enote{cf rem~\ref{rem:eric_note_01_Intro_5}}
%Cecilia Boschini, Jan Camenisch, and Gregory Neven present such a 
%framework for lattice-based anonymous credentials in their article 
%Relaxed Lattice-Based Signatures with Short Zero-Knowledge Proofs 
%accepted at ISC 2018. We have followed their framework to implement our 
%anonymous credential protocol.
%\enote{cf rem~\ref{rem:eric_note_01_Intro_6}}

%	Our code implements interactions involving anonymous credentials 
%between a user and a verifier. We used the framework from 
%~\cite{ISC:BosCamNev18} for our lattice-based anonymous credential 
%system, instantiated with the group signature from 
%\cite{CCS:delLyuSei18} and zero-knowledge arguments from 
%\cite{C:AttLyuSei20} (further improved in~\cite{C:LyuNguPla22}). The 
%zero-knowledge arguments from~\cite{C:AttLyuSei20} allowed us to use the 
%NTT and other optimizations in our implementation. We also used the 
%lattice estimator~\cite{DBLP:journals/jmc/AlbrechtPS15} to estimate the 
%security of the resulting anonymous credential protocol and choose the 
%parameters.

\iffalse
Our code implements interactions involving anonymous credentials between 
a user and a verifier. We used the framework from [BCN18]\footnote{C. 
Boschini, J. Camenisch, and G. Neven: Relaxed Lattice-Based Signatures 
with Short Zero-Knowledge Proofs. ISC 2018} for our lattice-based 
anonymous credential system, instantiated with the group signature from 
[dPLS18]\footnote{R. del Pino, V. Lyubashevsky, G. Seiler: Lattice-Based 
Group Signatures and Zero-Knowledge Proofs of Automorphism Stability. 
CCS 2018} and zero-knowledge arguments from [ALS20]\footnote{T. Attema, 
V. Lyubashevsky, G. Seiler: Practical Product Proofs for Lattice 
Commitments. CRYPTO 2020}. The zero-knowledge arguments from [ALS20] 
allowed us to use the NTT and other optimizations in our implementation. 
We also used the lattice estimator\footnote{
    Martin R. Albrecht, Rachel Player and Sam Scott. On the concrete 
hardness of Learning with Errors.
    Journal of Mathematical Cryptology. Volume 9, Issue 3, Pages 
169–203, ISSN (Online) 1862-2984,
    ISSN (Print) 1862-2976 DOI: 10.1515/jmc-2015-0016, October 2015
https://github.com/malb/lattice-estimator} to estimate the security of 
the resulting anonymous credential protocol and choose the parameters.
\fi

%As a result, our implementation of a post-quantum anonymous credential 
%is only an order of magnitude slower and bandwidth consuming than 
%previous anonymous credentials that are not post-quantum.

\iffalse
\subsection{Related Work}

The traditional way to prove one's identity on the Internet is to ask an 
authority to provide a credential, but usually without respecting the 
user's anonymity, making him be traced from server to server. Anonymous 
credentials, initially proposed by 
Chaum~\cite{DBLP:journals/cacm/Chaum85} allow the user to prove some 
properties on attributes without having to reveal them. Due to their 
practical interest, these schemes have received a lot of attention from 
the cryptographic community, for 
instance~\cite{10.5555/517876,EC:CamLys01,C:CamLys02,SCN:CamLys02,C:CamLys04,TCC:BCKL08,DBLP:conf/ima/IzabacheneLV11,NDSS:GarGreMie14,AC:CDHK15}. 
But to the best of our knowledge, no post-quantum scheme has been 
proposed so far.
\fi


\iffalse
\begin{verbatim}
    _.~"(_.~"(_.~"(_.~"(_.~"(

    Notes:
* reprendre le brouillon de Thomas et compléter

* intro generique d'AC + ajouter references AC non 
post-quantiques

* related work ? Cf commentaires ci-dessous

* directement AC (pas insister sur sig groupe), et implem

* CC: est-ce qu'il y a un avantage à utiliser 2018/779
+2020/517 plutôt que Vadim 2022/284 ? En termes d'implém ?

Éric: dans l'équation p3 on a un H(attribut) 
ils ne l'ont pas
eux ont un u fixé
nous c'est mieux car sinon on doit faire plusieurs 
signatures de groupe pour un AC 
-> nous, on peut mettre en une fois id et attr

on fait 2 signatures : sign(id | attr) et sign(id) 
et preuve même id

c'est une optimisation

Mais on pourrait le faire avec leur schéma, donc pas 
un avantage en fait... 
\end{verbatim}
\fi

\iffalse
La mani�re traditionnelle de prouver son identit� sur Internet consiste � 
deman- der une accr�ditation � une autorit�, souvent aux d�pens de l'anonymat 
de l'utili- sateur, qui peut au minimum �tre trac� de serveur en serveur. Les 
accr�ditations anonymes, initialement propos�es par Chaum [12], permettent 
d'�viter cet �cueil

en autorisant l'utilisateur � prouver des propri�t�s sur des attributs sans 
les r�v�- ler. En raison de leur int�r�t pratique, ces sch�mas ont re�u une 
forte attention de la communaut� cryptographique, par exemple [5, 7, 8, 9, 10, 
3, 15, 14, 6].

[3] Mira Belenkiy, Melissa Chase, Markulf Kohlweiss, and Anna Lysyanskaya.
P-signatures and noninteractive anonymous credentials. In Ran Canetti, edi-
tor, Theory of Cryptography, Fifth Theory of Cryptography Conference, TCC
2008, New York, USA, March 19-21, 2008., volume 4948 of Lecture Notes in
Computer Science, pages 356-374. Springer, 2008.

[5] Stefan A. Brands. Rethinking Public Key Infrastructures and Digital Certifi-
cates : Building in Privacy. MIT Press, Cambridge, MA, USA, 2000.

[6] Jan Camenisch, Maria Dubovitskaya, Kristiyan Haralambiev, and Markulf
Kohlweiss. Composable and modular anonymous credentials : Definitions
and practical constructions. In Tetsu Iwata and Jung Hee Cheon, editors,
Advances in Cryptology - ASIACRYPT 2015 - 21st International Conference on
the Theory and Application of Cryptology and Information Security, Auckland,
New Zealand, November 29 - December 3, 2015, Proceedings, Part II, volume
9453 of Lecture Notes in Computer Science, pages 262-288. Springer, 2015.

[7] Jan Camenisch and Anna Lysyanskaya. An efficient system for non-
transferable anonymous credentials with optional anonymity revocation. In
Birgit Pfitzmann, editor, Advances in Cryptology - EUROCRYPT 2001, Interna-
tional Conference on the Theory and Application of Cryptographic Techniques,
Innsbruck, Austria, May 6-10, 2001, Proceeding, volume 2045 of Lecture
Notes in Computer Science, pages 93-118. Springer, 2001.

[8] Jan Camenisch and Anna Lysyanskaya. Dynamic accumulators and applica-
tion to efficient revocation of anonymous credentials. In Moti Yung, editor,
Advances in Cryptology - CRYPTO 2002, 22nd Annual International Crypto-
logy Conference, Santa Barbara, California, USA, August 18-22, 2002, Pro-
ceedings, volume 2442 of Lecture Notes in Computer Science, pages 61-76.
Springer, 2002.

[9] Jan Camenisch and Anna Lysyanskaya. A signature scheme with efficient
protocols. In Stelvio Cimato, Clemente Galdi, and Giuseppe Persiano, edi-
tors, Security in Communication Networks, Third International Conference,
SCN 2002, Amalfi, Italy, September 11-13, 2002. Revised Papers, volume
2576 of Lecture Notes in Computer Science, pages 268-289. Springer, 2002.

[10] Jan Camenisch and Anna Lysyanskaya. Signature schemes and anonymous
credentials from bilinear maps. In Matthew K. Franklin, editor, Advances in
Cryptology - CRYPTO 2004, 24th Annual International CryptologyConference,
Santa Barbara, California, USA, August 15-19, 2004, Proceedings, volume
3152 of Lecture Notes in Computer Science, pages 56-72. Springer, 2004.

[14] Christina Garman, Matthew Green, and Ian Miers. Decentralized anony-
mous credentials. In 21st Annual Network and Distributed System Security
Symposium, NDSS 2014, San Diego, California, USA, February 23-26, 2014.
The Internet Society, 2014.

[15] Malika Izabach�ne, Beno�t Libert, and Damien Vergnaud. Block-wise p-
signatures and non-interactive anonymous credentials with efficient attri-
butes. In Liqun Chen, editor, Cryptography and Coding - 13th IMA Interna-
tional Conference, IMACC 2011, Oxford, UK, December 12-15, 2011. Procee-
dings, volume 7089 of Lecture Notes in Computer Science, pages 431-450.
Springer, 2011.


\fi
