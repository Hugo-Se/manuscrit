%%%%%%%%%%%%%%%%%%%%%%%%%%%%%%%%%%%%%%%%%%%%%%%%%%%%%%%%%%%%%%%%%%%%%%%%%%%%
%%%%%%%%%%%%%%%%%%%%%%%%%%%%%%%%%%%%%%%%%%%%%%%%%%%%%%%%%%%%%%%%%%%%%%%%%%%%
%%%%%%%%%%%%%%%%%%%%%%%%%%%%%%%%%%%%%%%%%%%%%%%%%%%%%%%%%%%%%%%%%%%%%%%%%%%%
\subsubsubsection{Notation}
Throughout the paper, we will write $\R = \ZZ[X]/(X^d + 1)$ for a $d$ that 
will be a power of two, in order for $X^d + 1$ to be cyclotomic. For a prime 
$q$, we will note $\R_q = \ZZ_q[X]/(X^d + 1)$.

We will work with the usual norms, 
	for $w \in \R$, $w = \sum_{i=1}^d w_i X^i$,
	\begin{align*}
		&\norm{w} = \sqrt{\sum_{i=0}^{d-1} |w_i|^2} \quad
		\norm{w}_1 = \sum_{i\in\llbracket 0, d-1 \rrbracket} |w_i| \quad
		\norm{w}_\infty = \max_{i\in\llbracket 0, d-1 \rrbracket} |w_i|
	\end{align*}
	for $\mathbf{w} \in \R^n$, $\mathbf{w} = (w_1, \cdots, w_n)$ we note
	\begin{align*}
	\norm{\mathbf{w}} = \sqrt{\sum_{i = 1}^n \norm{w_i}^2}\quad
	 \norm{\mathbf{w}}_{1} = \max_{i\in\llbracket 1,n \rrbracket} \norm{w_i}_{1} \quad
	 \norm{\mathbf{w}}_{\infty} = \max_{i\in\llbracket 1,n \rrbracket} \norm{w_i}_{\infty}
	\end{align*}

	We will also use the norms modulo multiples primes $u$, defined as the previous norms for
	the representative with coefficients taken in $\rrbracket-(u-1)/2, (u-1)/2\rrbracket$. The
	prime will be indicated as is $\norm{\cdot}_{u},\norm{\cdot}_{u,1},\norm{\cdot}_{u,\infty}$,
	unless if the context is clear enough to remove it.
	% For $w = \sum_{i=0}^{d-1} w_i X^i \in \mathcal{R}_q$ (or $\R$), we define the support of $w$, as
	% $\supp{w} = |\{i \in \llbracket 0, d-1\rrbracket\::\: w_i \neq 0\}|$
	% and for $\mathbf{w} = (w_1, \ldots, x_w) \in \R^n_q$ (or $\R^n$), we define the support of $\mathbf{w}$, as
	% $\supp{\mathbf{w}} = \sum_{i=1}^n \supp{w_i}$.
	% For $\mathbf{A} \in \R^{l\times k}$, we define $s_1(\mathbf{A}) = \max_{\norm{x}\neq0}\frac{\norm{\mathbf{A}x}}{\norm{x}}$.
%
	For $\beta > 0$, we will note $\ballRinfty{\beta} = \{ a \in \R : \norm{a}_\infty \leq \beta\}$.
%
	For a product of spaces $X \times Y$, we will note $\proj_X$ the projection on $X$.
	We write $x \sample X$ the uniform sampling of an element $x \in X$.

	For $\sigma > 0$, we denote by $\mathcal{D}_{\R , \sigma}$ the discrete Gaussian
	distribution on $\R$ with standard deviation $\sigma$. More precisely, each
	coordinate is taken by a discrete Gaussian distribution on $\ZZ$, of standard deviation~$\sigma$. The $\R$ will be omitted when it is clear from context.


%%%%%%%%%%%%%%%%%%%%%%%%%%%%%%%%%%%%%%%%%%%%%%%%%%%%%%%%%%%%%%%%%%%%%%%%%%%%
%%%%%%%%%%%%%%%%%%%%%%%%%%%%%%%%%%%%%%%%%%%%%%%%%%%%%%%%%%%%%%%%%%%%%%%%%%%%
%%%%%%%%%%%%%%%%%%%%%%%%%%%%%%%%%%%%%%%%%%%%%%%%%%%%%%%%%%%%%%%%%%%%%%%%%%%%
\subsubsubsection{Prime Splitting and Galois Automorphisms}
As in~\cite{C:AttLyuSei20}, we take $\tau$, $\theta$, $d$ powers of two such that $\tau \mid \theta \mid d$ and
we will consider primes numbers $\pcomI$, $\pcomII$, $\pencbig$ such that
\begin{equation}\label{eq:primes_rel_517}
	\forall u \in \{\pcomI, \pcomII, \pencbig\}\colon u - 1 \equiv 2 \theta \mod 4\theta.
\end{equation}

% We then define $\R_q = \ZZ_q[X]/(X^d + 1)$.
\cite{C:AttLyuSei20} shows that for each $u \in \{\pcomI, \pcomII, \pencbig\}$, there exist a
primitive $2\theta$ root of unity $\zeta_u \in {\ZZ}_u$.
Using the Chinese remainder theorem we know that there is a $\zeta \in \ZZ,\: \zeta \leq \pcomI\cdot \pcomII\cdot \pencbig$,
such that $\zeta \mod u = \zeta_u$ for $u\in\{\pcomI,\pcomII,\pencbig\}$.
Let $\II = \ZZ^\times_{2d}/\langle2\theta + 1\rangle$ and $\phi_i = 
X^{d/\theta} - \zeta^i$.

%	\begin{align*}
%		 & \II = \ZZ^\times_{2d}/\langle2\theta + 1\rangle \\
%		 & \phi_i = X^{d/\theta} - \zeta^i            \\
%	\end{align*}
	As shown in~\cite[Section2.2]{C:AttLyuSei20}, the $\phi_i$ form the irreducible decomposition of $X^d + 1$ modulo $u$:
		$X^d + 1 \mod q = \prod_{i\in \II}\phi_i$.
	
Following the work of~\cite{C:AttLyuSei20}, we will note:
	\begin{align*}
		 \Itau{\tau}= \ZZ^\times_{2\theta/\tau}\times \llbracket 0, \tau -
		 1\rrbracket \qquad &
		  \automorphismtau{\tau} := \automorphism_{2(\theta/\tau) + 1}                                   \\
		 \Phitau{\tau}_i = X^{\tau d / \theta} - \zeta^{i \tau} \qquad &
		  \phitau{\tau}_{i,j} = \automorphismtau{\tau}^j(X^{d/\theta} - \zeta^i)               \\
	\end{align*}

% We will now see a useful lemma that allow to ``stick'' multiples linear relations into one relation.
% \begin{proposition} \label{lemma:ctrcondition}
% 	Let $(q_l)_{i = 1}^v$ a finite number of distinct primes that all satisfy 
% 	$q_l - 1 \equiv 2 \theta \mod 4 \theta$,  a constant $B > 0$ and elements
% 		$\mathbf{A} \in \R^{x \times l}$, 
% 		$(\mathbf{z}_{i,j})_{(i, j) \in \Itau{\tau}} \subset \R^{l}$,
% 		$\mathbf{s} \in \R_{\pcomI}^x$ ,
% 		${(c_i)}_{i \in \ZZ^\times_{2\theta/\tau}} \in \R$ 
% 	such that:
% 	\begin{align*}
% 		&\forall (i,j) \in \Itau{\tau} \colon\: \mathbf{A} \mathbf{z}_{i,j} = \automorphismtau{\tau}^j\left(c_i\right) \mathbf{s}\!\!\!\mod \pcomI
% 		 \land \norm{\mathbf{z}_{i,j}} \leq B \\
% 		&\forall i \in \ZZ^{\times}_{2\theta/\tau}\colon\:  c_i \neq 0 \mod
% 		\Phitau{\tau}_i \land \norm{c_i}_1 \leq C \\
% 	\end{align*}
% 	Then, we can efficient construct a $\mathbf{z} \in \R^{l}$ such that either :
% 		      \begin{align*}
% 			                             & \mathbf{A} \mathbf{z} = \mathbf{s} \!\!\!\mod
% 						     \pcomI   \quad
% 			                             & \norm{\mathbf{z}}_{\infty} < \left(\prod_{l=1}^{v} q_l \right) /2   \\
% 			      \forall i,j\::\:\:\:\: & \automorphismtau{\tau}^j(c_i)\:\mathbf{z} =
% 			      \mathbf{z}_{i,j} \mod \left(\prod_{l=1}^{u} q_l \right)
% 		      \end{align*}
% 		      or $\mathbf{A} \mathbf{z} = 0 \land \norm{\mathbf{z}} < 2 B C$.
% \end{proposition}





%%%%%%%%%%%%%%%%%%%%%%%%%%%%%%%%%%%%%%%%%%%%%%%%%%%%%%%%%%%%%%%%%%%%%%%%%%%%
%%%%%%%%%%%%%%%%%%%%%%%%%%%%%%%%%%%%%%%%%%%%%%%%%%%%%%%%%%%%%%%%%%%%%%%%%%%%
%%%%%%%%%%%%%%%%%%%%%%%%%%%%%%%%%%%%%%%%%%%%%%%%%%%%%%%%%%%%%%%%%%%%%%%%%%%%
\subsubsubsection{MSIS, MLWE and NTRU problems}
We consider the same definitions than in~\cite{ISC:BosCamNev18}.
\begin{definition}
	The $\MSIS_{q,n,m,\beta}$ problem (over an implicit ring $\R$) is defined as
	follows. Given $\mathbf{A} \in \R_q^{n\times m}$ sampled uniformly at random. Find
	$\mathbf{s}\in\R^m$ such that $\mathbf{A}\mathbf{s} = 0$ and $0 < \norm{\mathbf{s}} \leq
	\beta$.
\end{definition}

\begin{definition}
	The decision $\MLWE_{q, m, n}$ problem (over an implicit ring $\R$) is defined as
	follows. Given $\mathbf{s}  \xleftarrow{\$} \ballRoneinfty^n$, let $A_{q,\mathbf{s}}$ the distribution
	obtained by sampling $\mathbf{a} \xleftarrow{\$} \R^n_q$, $e \xleftarrow{\$} \ballRoneinfty$,
	and returning $(\mathbf{a}, \left<\mathbf{a},\mathbf{s}\right> + e) \in \R^n_q \times
		\R_q$. The goal is to distinguish between $m$ samples from either $A_{q,
				\mathbf{s}}$ or the uniform distribution in $\R^n_q\times \R_q$.
\end{definition}

\begin{definition}
	The $\NTRU_{q,r}$ problem (over an implicit ring $\R$) is defined as
	follows.  The distribution A is defined by sampling ring elements $f, g \xleftarrow{\$} D_r$
	and outputting $h = f/g$ if $g$ is invertible in $\R_q$ (otherwise, re-sample g).
	The goal is to distinguish $h$ from a random element in $\R_q$.
\end{definition}


%%%%%%%%%%%%%%%%%%%%%%%%%%%%%%%%%%%%%%%%%%%%%%%%%%%%%%%%%%%%%%%%%%%%%%%%%%%%
%%%%%%%%%%%%%%%%%%%%%%%%%%%%%%%%%%%%%%%%%%%%%%%%%%%%%%%%%%%%%%%%%%%%%%%%%%%%
%%%%%%%%%%%%%%%%%%%%%%%%%%%%%%%%%%%%%%%%%%%%%%%%%%%%%%%%%%%%%%%%%%%%%%%%%%%%
\subsubsubsection{Relations, relaxed sets and relaxed relations}
	A relaxed set is a triplet
\begin{equation*}(X,\bar{X}, \projX)\end{equation*}
        where $X$, $\bar{X}$ are sets and $\projX$
	is a surjection $\bar{X} \rightarrow X$. We will sometimes call $\bar{X}$ the relaxation of
	$X$ and see $\projX$ as a function $\bar{X} \sqcup X \rightarrow X$ that is
	the identity on X.

	An element $\bar{x}$ will be called a relaxation of $x$ if 
$\projX(\bar{x}) = x$ or if
	$\bar{x} = x$. If $\bar{x} \neq x$, we will speak of 
strict relaxation.
	The elements of $\bar{X}$ will be called the relaxed elements.

	% Conversely, $x$ will be called a basis of $\bar{x}$.

	We define the language and the witness associated to a relation $\R$ as usual by setting
	$\Language{\R} = \{ x : \exists y, \R(x, y)\}$ and $\Witness{\R} = \{ y : \exists x, \R(x, y)\}$.

	For an element $x = (a,b) \in \relation$, we will write $a=\projL(x)$ its projection in
	$\Language{\relation}$ and $b=\projW(x)$ its projection in $\Witness{\R}$.

	A relaxed relation is a triplet $(\relation,\barrelation, \proj_R)$ where 
	$\relation$ and $\barrelation$ are relations and
	$(\Language{\relation},\Language{\barrelation}, \proj_R)$ is a relaxed set.
	We will call $\barrelation$ the relaxation of $\relation$.
	A relaxed witness of an element $x$ will be any witness of $\bar{x}$ such that
	$\bar{x}$ is a relaxation of $x$. We will sometime speak of strict witness if $\bar{x}$
	is a strict relaxation of $x$.
	% The elements of $\barrelation$ will sometimes be called the relaxed elements.


% sections suivantes -> annexe






\iffalse
    \begin{verbatim}
        _.~"(_.~"(_.~"(_.~"(_.~"(

Enfin juste la partie initiale definissant les AC,
le reste dans une annexe qui va sauter (pour l'avoir sous la main).

Ajouter l'annexe E, mais juste un blabla dans le corps, 
les figures dans une annexe qui va sauter. 
   
Olivier: J'ai juste mis le fichier initial, le ajouter mais non 
mais si c'est pas clair :p

\end{verbatim}
\fi

