%%%%%%%%%%%%%%%%%%%%%%%%%%%%%%%%%%%%%%%%%%%%%%%%%%%%%%%%%%%%%%%%%%%%%%%%%%%%
%%%%%%%%%%%%%%%%%%%%%%%%%%%%%%%%%%%%%%%%%%%%%%%%%%%%%%%%%%%%%%%%%%%%%%%%%%%%
%%%%%%%%%%%%%%%%%%%%%%%%%%%%%%%%%%%%%%%%%%%%%%%%%%%%%%%%%%%%%%%%%%%%%%%%%%%%
\subsection{Definition}\label{sec:AC:definition}


An \textit{anonymous credential scheme} for a set of identities $\SetIdAC$ and 
a set of attributes $\SetAttrAC$ is a tuple of algorithms $(\ParGen, 
\IKeyGen,\allowbreak \OKeyGen, \Issue, \Present, \Open, 
\VerifyCred,\allowbreak \VerifyPt, \ExtractAttr)$:

\begin{compactitem} 
\item The \textbf{parameter generation algorithm $\bm{\ParGen}$}
	returns the public parameters $\acpar$.

\item The \textbf{issuer key generation algorithm $\bm{\IKeyGen(\acpar)}$}
returns an issuer (public, secret) key pair $(\ipk, \isk)$.

\item The \textbf{opener key generation algorithm $\bm{\OKeyGen(\acpar, \ipk)}$}
returns an opener (public, secret) key pair $(\opk, \osk)$.
% \enote{cf rem~\ref{rem:eric_note_03_Prelim_1}}; 

\item The \textbf{credential issuance algorithm $\bm{\Issue(\isk, \identity, 
(\attralphaR))}$} creates a credential $\cred$ associated to an identity 
$\identity$ and a set of attributes $\attralphaR$. It is possible to recover 
the set of attribute $\attralphaR$ from $\cred$ computing 
$\ExtractAttr(\cred)$. We call $\attralphaR$ the attributes associated to 
$\cred$.

\item The \textbf{presentation algorithm $\bm{\Present(\acpar, \ipk, \opk, \cred, S)}$} creates a presentation token $\pt$ using a credential $\cred$ with associated attributes $\attralphaR$ and $S \subset R$. It is 
	possible to recover the subset of attributes $(\alpha_i)_{i \in S}$ from $\pt$ computing 
$\ExtractAttr(\pt)$. We call $(\alpha_i)_{i \in S}$ the attributes associated to $\pt$.

\item The \textbf{opening algorithm $\bm{\Open(\acpar, \ipk, \osk, \pt)}$} returns the 
identity of the credential used to create the presentation (or~$\bot$).

%. This algorithm can also return $\bot$.

\item The \textbf{credential verification algorithm 
$\bm{\VerifyCred(\acpar,\allowbreak \ipk, \cred)}$} checks the validity of the 
credential for the associated attributes $(\alpha_i)_{i \in R}$ it contains and returns $1$ (if valid) or $0$.

\item The \textbf{presentation token verification algorithm 
$\bm{\VerifyPt(\acpar,\allowbreak \ipk, \opk, \pt)}$} checks the validity of the 
presentation token for the associated attributes $(\alpha_i)_{i \in S}$ it contains and 
returns $1$ (if valid) or~$0$. \end{compactitem}

We consider the following security properties (formally stated 
in~\cite[Appendix G.1]{ISC:BosCamNev18})

\begin{compactitem}

\item \textbf{Anonymity:} A user cannot distinguish from which credential a 
presentation is made.

\item \textbf{Unforgeability:} A user cannot create a valid presentation 
token without using a credential or restraining a valid presentation 
token to a smaller subset of attributes.

\item \textbf{Traceability:} All presentations should trace to an issuer.
If computed by a collusion of users, it should trace to a member of this 
collusion.

\end{compactitem}

%\enote{cf rem~\ref{rem:eric_note_03_Prelim_2}}



%%%%%%%%%%%%%%%%%%%%%%%%%%%%%%%%%%%%%%%%%%%%%%%%%%%%%%%%%%%%%%%%%%%%%%%%%%%%
%%%%%%%%%%%%%%%%%%%%%%%%%%%%%%%%%%%%%%%%%%%%%%%%%%%%%%%%%%%%%%%%%%%%%%%%%%%%
%%%%%%%%%%%%%%%%%%%%%%%%%%%%%%%%%%%%%%%%%%%%%%%%%%%%%%%%%%%%%%%%%%%%%%%%%%%%
\subsection{Construction}

Our anonymous credential scheme builds upon a relaxed signature scheme and a 
relaxed verifiable encryption scheme that are compatible, along with 
associated (relaxed) NIZK proofs. We state in the following subsections our definitions 
for such schemes before presenting our generic scheme.

%of relaxed  signature scheme and relaxed verifiable encryption.

%%%%%%%%%%%%%%%%%%%%%%%%%%%%%%%%%%%%%%%%%%%%%%%%%%%%%%%%%%%%%%%%%%%%%%%%%%%%
%%%%%%%%%%%%%%%%%%%%%%%%%%%%%%%%%%%%%%%%%%%%%%%%%%%%%%%%%%%%%%%%%%%%%%%%%%%%
%%%%%%%%%%%%%%%%%%%%%%%%%%%%%%%%%%%%%%%%%%%%%%%%%%%%%%%%%%%%%%%%%%%%%%%%%%%%
\subsubsubsection{Relaxed Signature 
Scheme}\label{sec:relaxed_signature_scheme}
	We use the same definition as in~\cite[Section 5.1]{ISC:BosCamNev18}. The 
only difference is our different definition of a relaxed set: instead of using 
a function $g: X \rightarrow 2^{\bar{X}}$, we consider a surjection $\bar{X} 
\rightarrow X$. One can easily switch from one vision to the other.

A relaxed signature scheme  associated with a relaxed message 
space $(\SetMessSign,\SetRelMessSign, \projsign)$ consists of 
the following algorithms:

%a relaxed space $(\SetMessSign, \projsign, \SetRelMessSign)$ and 

\begin{compactitem}
	\item The \textbf{key generation algorithm $\bm{\SignKeyGen()}$} returns a 
couple of (public, secret) signature keys $(\ssk, \spk)$.
	\item The \textbf{signature algorithm $\bm{\Sign(\ssk, m)}$} computes the 
signature $\signature$ of a	message $m$ with secret key $\ssk$.
	\item The \textbf{verification algorithm $\bm{\SignVf(\ssk, \signature, 
m)}$} verifies if $\signature$ is a valid signature of $m \in \SetMessSign 
\sqcup \SetRelMessSign$ for the public key $\spk$. It returns $1$ if it is 
valid and $0$ otherwise.
\end{compactitem}

The idea is to create signatures for messages, but verify both messages and 
relaxed messages.
	 
In the definition of the forgery game, we forbid the adversary to send a 
signature of a relaxation of a message it obtained the signature from the signature oracle. The 
detailed game is described in~\cite[Section 5.1]{ISC:BosCamNev18}.


%%%%%%%%%%%%%%%%%%%%%%%%%%%%%%%%%%%%%%%%%%%%%%%%%%%%%%%%%%%%%%%%%%%%%%%%%%%%
%%%%%%%%%%%%%%%%%%%%%%%%%%%%%%%%%%%%%%%%%%%%%%%%%%%%%%%%%%%%%%%%%%%%%%%%%%%%
%%%%%%%%%%%%%%%%%%%%%%%%%%%%%%%%%%%%%%%%%%%%%%%%%%%%%%%%%%%%%%%%%%%%%%%%%%%%
\subsubsubsection{Relaxed NIZK proofs (in the random oracle model)}
	Our general scheme uses the notion of non-interactive zero-knowledge 
(NIZK) proof system for a relaxed relation $(\relation,\barrelation, \proj_R)$.
Their definition is a slight adaption 
 of the one stated in~\cite[Section 3.1]{ISC:BosCamNev18}. It is 
important to note that such a protocol
only allows to compute proofs for elements $x \in \Language{\relation}$, 
when a witness $w$ is known,
 and that the relaxed special soundness property applied for proofs of 
$x$ only allows to extract a
 relaxed witness $\bar{w}$ of $x$, so that $(x, \bar{w}) \in 
\barrelation$.

% "Pour des raisons de place, on laisse le lecteur interessé voir~\cite[Section
% 3.1]{ISC:BosCamNev18}"

As a side note, the challenges for the NIZK proofs are not chosen 
uniformly but according to a distribution $\distchall{\rho}$ (see
 Table~\ref{tab:parameters_ac_scheme}), which is thus also the case for the
hash function $\hashZK$.

% \enote{traduire}Notons que pour les preuves NIZK les challenges ne sont 
%pas choisis uniforméments dans
% $\SetChallenges$ mais via une distribution $\distchall{\rho}$ (voir
% Table~\ref{tab:parameters_ac_scheme}). La fonction de hash $\hashZK$ 
%utilisée est donc soumise �
% l'hypothèse ROM avec cette distribution plutôt que la distribution 
%uniforme.

% and refer the interested reader to~\cite{ISC:BosCamNev18} for details. 


%%%%%%%%%%%%%%%%%%%%%%%%%%%%%%%%%%%%%%%%%%%%%%%%%%%%%%%%%%%%%%%%%%%%%%%%%%%% 
%%%%%%%%%%%%%%%%%%%%%%%%%%%%%%%%%%%%%%%%%%%%%%%%%%%%%%%%%%%%%%%%%%%%%%%%%%%% 
%%%%%%%%%%%%%%%%%%%%%%%%%%%%%%%%%%%%%%%%%%%%%%%%%%%%%%%%%%%%%%%%%%%%%%%%%%%% 
\subsubsubsection{Relaxed Verifiable 
Encryption}\label{sec:relaxed_verifiable_encryption} Even if our 
instantiation makes use of the verifiable encryption 
described in~\cite{EC:LyuNev17}, our definition of a verifiable encryption is 
slightly different and its instantiation  actually uses both the 
verifiable encryption of~\cite{EC:LyuNev17} and the commitment scheme 
of~\cite{C:AttLyuSei20}. This is explained by the fact that a 
verifiable encryption does not only encrypt an element, but a relation 
fulfilled by the element.

%our use of a different 
%definition of a verifiable encryption, that emphasizes the fact that a 
%verifiable encryption doesn't only encrypt an element, but a relation 
%"dans laquelle il est"\enote{..}.

We consider a relaxed space of clear messages $(\SetMessEnc, 
\SetRelMessEnc,\allowbreak \projenc)$ and probabilistic algorithms $(\EncKeyGen, 
\Enc, \Dec)$ such that
	\begin{compactitem}

		\item $\EncKeyGen$ outputs a couple of (public, secret)
keys $(\epk, \esk) \in \SetPublicKeysEnc \times \SetSecretKeysEnc$.

		 \item For $(\epk, \mess) \in \SetPublicKeysEnc \times 
\SetMessEnc$, $\Enc(\epk, \mess)$ outputs a ciphertext $\cipher \in 
\SetCipherEnc$.

		 \item For a ciphertext $\cipher \in \SetCipherEnc$ and a 
secret key $\esk \in \SetSecretKeysEnc$, $\Dec(\esk, \cipher)$ outputs a 
message $\mess \in \SetMessEnc \sqcup \SetRelMessEnc \sqcup \{\bot\}$. 
We say that the decryption fails if $\mess = \bot$.

	\end{compactitem}

	We require that $(\EncKeyGen \Enc, \Dec)$ is a IND-CPA 
encryption scheme with message space $\SetMessEnc$ and ciphertext space 
$\SetCipherEnc$.

Moreover, we consider 
	\begin{compactitem}
		\item two relaxed relations $(\relationclear, \barrelationclear,
\projclear)$, $(\relationcipher, \barrelationcipher,\allowbreak \projcipher)$ 
			with:
		$\Language{\relationclear} = \SetMessEnc \times \SetPubEnc$,
		$\Language{\barrelationclear} = \SetRelMessEnc \times \SetPubEnc$, $\projclear =
	(\idfunction, \projenc)$. 
		$\Language{\relationcipher} = \Language{\barrelationcipher} = \SetPublicKeysEnc \times \SetCipherEnc \times
	\SetPubEnc$ and $\projcipher = \idfunction$.

	\item a NIZK proof system for the relaxed relation 
$(\relationcipher, \barrelationcipher,\allowbreak \projcipher)$,

%\enote{a definir?} 

 \item a subset $\DomEncRel \subset \SetPublicKeysEnc \times \relationclear$,

\item a probabilistic algorithm $\EncRel : \DomEncRel \rightarrow 
\relationcipher$ that extends the function $\Enc$ to relation thanks to 
the requisite $\projL \circ \EncRel = (\Enc, \id) \circ (\id, \projL)$,

\item a relaxed extraction of relation function $\barextractionrelation 
:\barrelationcipher \rightarrow \barrelationclear$ such that the 
following diagram commutes:

\[ \begin{tikzcd}
	\barrelationcipher \arrow{rr}{\barextractionrelation}
	\arrow[swap]{drr}{\proj_{\SetPubEnc}}
	 &&\barrelationclear \arrow{d}{\proj_{\SetPubEnc}} \\
	&& \SetPubEnc\end{tikzcd} \]
\end{compactitem}
	Finally, we ask the following property, that expresses
	the compatibility of the decryption algorithm and of the definition of the witness of
	$\barrelationcipher$.
	\begin{compactitem}
	\item For $(\epk, \esk) \xleftarrow{\$} \EncKeyGen, \cipher \in \SetCipherEnc$ and for all $X \in \barrelationcipher$ such
		that \[\epk = \proj[\SetPublicKeysEnc](\projL(X)),\:\: \cipher = \proj_{\SetCipherEnc}(\projL(X))\]
	the probability to not have
	\[ \projenc\big(\Dec(\esk, \cipher)\big) = \projenc(\proj_{\SetRelMessEnc}\big(\barextractionrelation(X)\big))\]
	is negligible. Here, we take the convention that $\projenc(m) = m$ if $m \in \SetMessEnc$.
\end{compactitem}


%%%%%%%%%%%%%%%%%%%%%%%%%%%%%%%%%%%%%%%%%%%%%%%%%%%%%%%%%%%%%%%%%%%%%%%%%%%%
%%%%%%%%%%%%%%%%%%%%%%%%%%%%%%%%%%%%%%%%%%%%%%%%%%%%%%%%%%%%%%%%%%%%%%%%%%%%
%%%%%%%%%%%%%%%%%%%%%%%%%%%%%%%%%%%%%%%%%%%%%%%%%%%%%%%%%%%%%%%%%%%%%%%%%%%%
\subsubsubsection{Anonymous Credential}\label{sec:general_AC}
Our generic construction will need to use relaxed signature and relaxed verifiable encryption schemes
with the following requirements:
\begin{compactitem}
	\item $\SetMessSign = \SetMessEnc \times \SetAttrAC$, $\SetRelMessSign = \SetRelMessEnc \times \SetAttrAC$
	\item The relaxed function of the signature scheme preserves the attributes : $\proj{\SetAttrAC} \circ \projsign = \proj{\SetAttrAC}$.
	\item The relations of the relaxed verifiable encryption are such that :
%		\tiny
		  \begin{align*}
\relationclear    & := \Big\{\left(\left(\identity, \spk, R, 
{(\alpha_i)}_{i \in R}\right),\left({(\sigma_i)}_{i \in R}\right)\right)\::\:
\hspace{0.1cm} \\ &\forall i \in R,\, \SignVf(\spk, (\identity,\alpha_i),
\signature_i) = 1 \Big\}                                                                                                                   \\ \barrelationclear & := \Big\{\left(\left(\baridentity, \spk, R, {(\alpha_i)}_{i \in R}\right),
\left({(\baridentity_i)}_{i \in R}, {(\signature_i)}_{i \in R}\right)\right)\::\:
\hspace{0.1cm} \\ &\forall i \in R, 
\SignVf(\spk, (\identity,\alpha_i), \signature_i) = 1 \\
%\hphantom{\barrelationclear  := \Big\{\left(\left(\baridentity, \spk, R, {(\alpha_i)}_{i \in R}\right),
%\left({(\baridentity_i)}_{i \in R}, {(\signature_i)}_{i \in R}\right)\right)\::\:
%\hspace{0.1cm}} \\
& \hspace{7mm}\wedge \projsign(\baridentity_i) = \projsign(\baridentity) \Big\}
      \end{align*}
      \normalsize
	\item For simplicity, we suppose that  $\DomEncRel = \SetPublicKeysEnc
		\times \relationclear$. In the
		practical instantiation of Section~\ref{sec:Instanciation}, the keygens for Issuer and
		Opener will be made such that the keys $\ipk$ and $\opk$ share an element (this is why
		 $\OKeyGen$ have $\ipk$ in input),
		$\DomEncRel$ will thus only contain the couples such that $\ipk$ and $\opk$ contain this element.
\end{compactitem}

%\todoC{pas compris la parenthese}

The complete scheme can be found in Figure~\ref{fig:general_construction_AC}.
\begin{figure*}
\begin{pcvstack}[center]
	\begin{pchstack}
		\procedure[headlinesep=2pt,bodylinesep=2pt]{$\ParGen()$}{%
			\acpar \gets \bot \\
			\pcreturn \acpar
		}
		\procedure[headlinesep=2pt,bodylinesep=2pt]{$\IKeyGen(\acpar)$}{%
			(\spk, \ssk) \gets \SignKeyGen()\\
			\ipk \gets \spk \\
			\isk \gets \ssk \\
			\pcreturn (\ipk, \isk)
		}
		\procedure[headlinesep=2pt,bodylinesep=2pt]{$\OKeyGen(\acpar, \ipk)$}{%
			(\epk, \esk) \gets \EncKeyGen() \\
			\opk \gets \epk \\
			\osk \gets \esk \\
			\pcreturn (\opk, \osk)
		}
		\procedure[headlinesep=2pt,bodylinesep=2pt]{$\Issue(\isk, \identity,\attralphaR)$}{%
			\pcfor i \in R \colon\\
			\t\signature_{i} \gets \Sign\big(\ipk,(\identity, \alpha_i\sqcup i)\big) \\
			\cred \gets \left(\identity,{(\alpha_i)}_{i \in R},{(\signature_{i})}_{i \in R}\right) \\
			\pcreturn \cred
		}
	\end{pchstack}
	\pcvspace
	\begin{pchstack}
		\procedure[headlinesep=2pt,bodylinesep=2pt]{$\Present\big(\acpar, \ipk,\opk, \cred = \left(\identity,{(\alpha_i)}_{i \in R},{(\signature_{i})}_{i \in R}\right), S\big)$}{%
			x \gets \left(\left(\identity, \spk, S, {(\alpha_i)}_{i \in  S}\right),\left({(\sigma_i)}_{i \in S}\right)\right) \\
			\pccomment{$(\epk, x) \in \DomEncRel$ by hypothesis} \\
			C = \EncRel(\epk, x) \\
			\text{creates a NIZK proof $\nizkproof$ for $C$}\\
			\pcreturn  ({(\alpha_i \sqcup i)}_{i \in S}, \projL(C), \pi)
		}
		\procedure[headlinesep=2pt,bodylinesep=2pt]{$\Open\big(\acpar, \ipk,\osk, \pt\big)$}{%
			c \dgets \text{the cipher included in the presentation } \pt \\
			\identity \gets \Dec(\esk, c)\\
			\pcreturn \identity
		}
	\end{pchstack}
	\pcvspace
	\begin{pchstack}
		\procedure[headlinesep=2pt,bodylinesep=2pt]{$\VerifyCred\big(\acpar, \ipk, \cred = \left(\identity,{(\alpha_i)}_{i \in R},{(\signature_{i})}_{i \in R} \right)\big)$}{%
			\pcfor i \in R \colon \\
			\t \pcif \SignVf(\spk, \signature_i, \alpha_i \sqcup i) = 0 \pcthen \pcreturn 0 \\
			\pcreturn 1
		}
		\pchspace[2em]
		\procedure[headlinesep=2pt,bodylinesep=2pt]{$\VerifyPt\big(\acpar, \ipk, \opk, \pt)$}{%
			\text{Verify the proof $\nizkproof$} \\
			\pcif \text{the proof is valid} \pcthen \pcreturn 1 \\
			\pcelse \pcreturn 0
		}
	\end{pchstack}
\end{pcvstack}
\caption{\label{fig:general_construction_AC}General construction of Anonymous Credentials}
\end{figure*}

%%%%%%%%%%%%%%%%%%%%%%%%%%%%%%%%%%%%%%%%%%%%%%%%%%%%%%%%%%%%%%%%%%%%%%%%%%%%
%%%%%%%%%%%%%%%%%%%%%%%%%%%%%%%%%%%%%%%%%%%%%%%%%%%%%%%%%%%%%%%%%%%%%%%%%%%%
%%%%%%%%%%%%%%%%%%%%%%%%%%%%%%%%%%%%%%%%%%%%%%%%%%%%%%%%%%%%%%%%%%%%%%%%%%%%
\subsection{Sketch of proofs}\label{sec:sketch_proof_secu_AC}

We now informally show why this is a secure anonymous credential scheme, 
for each of the security properties explained in 
Section~\ref{sec:AC:definition} and that are formally defined in 
\cite[Appendix G.1]{ISC:BosCamNev18}.

%\label{sec:preliminaries}

\begin{compactitem} 
\item \textbf{Anonymity:} We can replace the NIZK 
proofs by simulations and the ciphertext by the encryption of any
	fixed clear message. This shows that
	$\Adv{\ProbAnomCredAC}{\adv} \leq \Adv{\ProbCPAEnc}{\adv} + 
\Adv{\ProbNIZKAC}{\adv}$
	where $\ProbAnomCredAC$ is the anonymity game of the anonymous 
credential, $\ProbCPAEnc$ the
	IND-CPA game of the encryption scheme $(\EncKeyGen, \Enc, \Dec)$ 
and $\ProbNIZKAC$ the game of distinction of a NIZK proof and its 
simulation.

\item \textbf{Unforgeability:} If a user can forge with a sufficiently good 
advantage an anonymous
	credential, one can extract a witness from the proof and use the 
function
	$\barextractionrelation$ to obtain an element 
$\left(\left(\baridentity, \spk, R, {(\alpha_i)}_{i \in R}\right),
		      \left({(\baridentity_i)}_{i \in R}, 
{(\signature_i)}_{i \in R}\right)\right)
	      \in \barrelationclear$. The conditions for winning of the 
game then imply that one of
	      the signatures $\signature_i$ is a forgery for the relaxed 
signature scheme.

%\enote{clair?}

      \item \textbf{Traceability:} If a user can forge with a sufficiently good 
advantage an anonymous credential, opened in an element 
$\tilde{\identity}$ that is not the relaxation of the identity of the 
collusion (note that $\tilde{\identity}$ can even be equal to 
$\bot$), one can extract a witness from the proof and use
the function
	$\barextractionrelation$ to obtain an element 
$\left(\left(\baridentity, \spk, R, {(\alpha_i)}_{i \in R}\right),
		      \left({(\baridentity_i)}_{i \in R}, 
{(\signature_i)}_{i \in R}\right)\right)
	      \in \barrelationclear$.  Moreover, the compatibility of 
the decryption algorithm and the definition of the witness of
	$\barrelationcipher$ implies that $\tilde{\identity}$ is a 
relaxation of $\baridentity$ (in particular, it is not equal to $\bot$).
	      This means that each $\baridentity_i$ is a relaxation of 
an identity that is not
	      in the collusion. The conditions for winning of the game then
imply that one of the
	      signatures $\signature_i$ is a forgery of the relaxed 
signature scheme. \end{compactitem}


\iffalse
\begin{verbatim}
    _.~"(_.~"(_.~"(_.~"(_.~"(

    copier-coller depuis new_version_article le début de la section 4, avant 4.1 ?
copier-coller l'annexe K dans une annexe qui va sauter ?

Notes:

* Éric m'a parlé de l'annexe I page 55, mais pas la place ? 

Olivier: Mis le main... le reste bleh
\end{verbatim}
\fi
