\newtheorem{remark}{Remark}

% General Commands
\mathchardef\mhyphen="2D % Define a "math hyphen"

% Common operations/functions
\providecommand{\gcd}{\mathsf{gcd}} 


\newcommand{\proj}[1][]{
\ifstrempty{#1}{\mathsf{proj}}{\mathsf{proj}_{#1}}}

% for nint
\DeclarePairedDelimiter{\nint}\lfloor\rceil

% Linear algebra
\providecommand{\innerproduct}[2]{\left\langle #1, #2 \right\rangle} 
\renewcommand{\Im}{\mathsf{Im}} % Image of a fonction
\renewcommand{\vec}[1]{\mathbf{\lowercase{#1}}} % for vectors of the ring. Lower-case!
\newcommand{\mat}[1]{\mathbf{\uppercase{#1}}} % For matrix of the ring. Upper-case!
\newcommand{\Ker}{\mathsf{Ker}} % Ker of a linear function 
\newcommand{\Span}{\mathsf{Span}} % Linear Span of a subset of a vector space
\newcommand{\Id}{\mathsf{I}}

% Common sets
\providecommand{\RR}{\mathbb{R}}
\providecommand{\RRplus}{\mathbb{R}_{\geq 0}}
\providecommand{\RRplusstar}{\mathbb{R}^*_{\geq 0}}
\providecommand{\ZZ}{\mathbb{Z}}
\providecommand{\ZZq}{\mathbb{Z}}
\providecommand{\ZZplus}{\ZZ_{\geq 0}}
\providecommand{\ZZplusstar}{\ZZplus^*}
\providecommand{\NN}{\mathbb{N}}
\providecommand{\NNstar}{\mathbb{N}^*}
\providecommand{\QQ}{\mathbb{Q}}

% a short equal sign. 
\newcommand{\shorteq}{\hstretch{0.8}{=}}
\newcommand{\shortminus}{\hstretch{0.8}{-}}
\newcommand{\shortplus}{\hstretch{0.9}{+}}
\newcommand{\shortldots}{\hstretch{0.85}{\ldots}}

% common norms/abs values/card
\providecommand{\abs}[1]{\left\lvert #1 \right\rvert} % absolute value 
\providecommand{\norm}[1]{\left\lVert #1 \right\rVert} % euclidean norm
\newcommand{\onenorm}[1]{\norm{#1}_1} % l1-norm
\newcommand{\infnorm}[1]{\norm{#1}_\infty}
\newcommand{\singularnorm}[1]{\ifstrempty{#1}
				{\mathsf{s}_1}
				{\mathsf{s}_1\!\left(#1\right)}
			} % the norm of a matrix (the smaller |A|
			% s.t |Au|_2 \leq |A||u|_2 for each u
\newcommand{\card}[1]{{\left|#1\right|}}

\newcommand{\bnorm}[1]{{|#1|}_{\textsf{bin}}} % A suppr?



% a modifed proof environment
\newenvironment{modifproof}[2][]{% version of proof where "proof" can be modified to another text.
				 % It is compatible with the optional option of proof that write
				 % text into parenthesis.
\let\oldproofnameformodifproof\undefined
\newcommand{\oldproofnameformodifproof}{\proofname}
\renewcommand{\proofname}{\textbf{#2}}
\ifstrempty{#1}{\proof}{\proof[#1]}}
{\endproof 
\renewcommand{\proofname}{\oldproofnameformodifproof}}


%
% new \oset macro: from https://tex.stackexchange.com/questions/194798/change-vertical-space-in-overset
% Example X_n\oset[.45ex]{\text{rth}}{\to} X 
% with removed mathrel
\makeatletter
\renewcommand{\oset}[3][0ex]{%
  \mathop{#3}\limits^{
    \vbox to#1{\kern-2\ex@
    \hbox{$\scriptstyle#2$}\vss}}}
\makeatother

%new \usetmacro: from https://tex.stackexchange.com/questions/194798/change-vertical-space-in-overset
% with removed mathrel
\makeatletter
\newcommand{\uset}[3][0ex]{%
  \mathop{#3}\limits_{
    \vbox to#1{\kern-7\ex@
    \hbox{$\scriptstyle#2$}\vss}}}
\makeatother

% See https://www.tug.org/TUGboat/Articles/tb22-4/tb72perlS.pdf to understand boxes
% a left align uset and underset, see https://tex.stackexchange.com/questions/671746/horizontal-alignement-leftalign-kind-with-underset-overset
\newcommand{\lunderset}[2]{\underset{\mathrlap{#1}}{\mathrlap{#2}}\phantom{#2}}
\newcommand{\luset}[3][0ex]{\uset[#1]{\mathrlap{#2}}{\mathrlap{#3}}\phantom{#3}}
