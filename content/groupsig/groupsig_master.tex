{

%\subimport{./}{preamble/packages.tex}
\subimport{./}{preamble/meta.tex}
\subimport{./}{preamble/common.tex}
\subimport{./}{preamble/crypto_common.tex}
\subimport{./}{preamble/math_common.tex}
\subimport{./}{preamble/math_probabilities.tex}
\subimport{./}{preamble/math_lattices.tex}
\subimport{./}{preamble/specific_notations.tex}
\subimport{./}{preamble/ARES.tex}


\chapterquote[Efficient Implementation of a PQ Anonymous Credential Protocol]{Efficient Implementation of a Post-Quantum Anonymous Credential Protocol}{« Three may keep a Secret, if two of them are dead. »\footnote{~« Trois personnes peuvent garder un secret, si deux d’entre elles sont mortes. »}}{Poor Richard’s Almanack}{Benjamin Franklin} \label{cha:groupsig}


\enluminure{A}{uthentication on the Internet} usually has the drawback of 
leaking the identity of the users, or at least allowing to trace them from a 
server to another. Anonymous credentials overcome this issue, by allowing 
users to reveal the attributes necessary for the authentication, without 
revealing any other information (in particular not their identity). In this 
chapter, we provide a generic framework to construct anonymous credential 
schemes and use it to give a concrete construction of post-quantum 
(lattice-based) anonymous credential protocol. Our protocol thus allows for 
long-term security even when one considers the emergence of quantum computers 
able to break widely used traditional computational assumptions, such as RSA, 
the discrete logarithm or Diffie-Hellman. We also give a concrete 
implementation of our protocol, which is only one order of magnitude slower 
and bandwidth consuming than previous anonymous credentials that are not 
post-quantum.

\section{Introduction}\label{sec:introduction}
\subimport{./}{01_Intro.tex}

\section{Technical Overview}\label{sec:technical_details}
\subimport{./}{02_TechIntr.tex}

\section{Preliminaries}\label{sec:preliminaries}
\subimport{./}{03_Prelim.tex}

\section{Anonymous Credential Scheme}\label{sec:anonymous_credential_scheme}
\subimport{./}{04_ACS.tex}

\section{Instantiation and Parameters}\label{sec:instanciation_parameters}
\label{sec:Instanciation}
\subimport{./}{05_Instant.tex}

\section{Implementation}\label{sec:implementation}
\subimport{./}{06_Implem.tex}

\iffalse


\fi

}
