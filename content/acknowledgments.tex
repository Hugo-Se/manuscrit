% !TEX root = ../master.tex


\chapter*{Remerciements} % (fold)
\label{cha:acknowledgements}
\addstarredchapter{\nameref{cha:acknowledgements}Remerciements} 
%\addcontentsline{toc}{chapter}{Remerciements}

%Je vous préviens, cette page ne remerciements va vous ennuyer profondément.
%Ne vous avais-je pas prévenu ?

% d’une main de maître (de conférence)

% dont l’assistance, quelle fut directe ou indircete, m’a toujours

À tous seigneurs tous honneurs : je tiens à commencer ces quelques lignes de remerciements en exprimant la gratitude que j’ai pour Céline Chevalier et Thomas Ricosset, qui furent pendant plus de trois ans mes directeurs de thèse.\par

Céline Chevalier, avec la bienveillance qu’elle a constamment manifestée à mon égard, s’est toujours empressée de m’assurer de sa disponibilité pour répondre à la moindre de mes questions et j’ai pu ainsi profiter tant de ses conseils scientifiques éclairés que de son aide pour affronter les problèmes pratiques ou administratifs qui se sont présentés.
Pour cela et pour tout le reste, je lui adresse mes plus chaleureux remerciements.\par

Un grand merci également à Thomas Ricosset qui, par sa conduite avisée de mes travaux et ses remarques judicieuses sur ceux-ci, a fortement contribué à faire de la préparation de cette thèse l’expérience, des plus formatrices, qu’elle a été.\par

Je ne veux bien sûr pas manquer de remercier vivement les membres de mon jury de thèse pour avoir accepté d’en faire partie et de prendre de leur temps pour remplir comme ils l’ont fait le rôle de juré avec une diligence qui s’est reflétée dans la qualité des questions qui m’ont été posées lors de ma soutenance.
Il s’agit, outre de mes directeurs de thèse, d’Olivier Blazy, de Dương Hiệu Phan, d’Adeline Roux-Langlois, de Damien Vergnaud et du grand manitou du département d’informatique de l’École normale supérieure, David Pointcheval, dont l’assistance et les instructions m’ont été plus que précieuses au cours de ces dernières années dans le laboratoire qu’il dirige.\par

Je suis tout particulièrement reconnaissant aux rapporteurs, Adeline Roux-Langlois et Dương Hiệu Phan, qui par ma faute ont dû s’atteler à examiner mon mémoire en un temps restreint en plein milieu de l’été alors qu’à n’en pas douter ils se rêvaient à ce moment sur quelque plage paradisiaque d’une île du Pacifique... Je leur sais gré de l’effort auquel ils ont consenti ; je souhaiterais qu’ils sachent que j’ai apprécié le soin manifeste qu’ils ont mis dans la lecture critique de mon manuscrit et la rédaction des comptes rendus.\par

Merci aux membres du laboratoire de l’École, actuels et anciens, étudiants et chercheurs, dont j’ai pu croiser la route ou avec lesquels j’ai pu travailler pour le très bon accueil qu’ils m’ont fait dans l’équipe et pour tous les moments privilégiés d’échange intellectuel qui ont contribué à rendre agréable mes recherches en leur compagnie. Qu’on me pardonne de n’en citer que quelques uns : Michel Abdalla, Léonard Assouline, Hugo Beguinet, Wissam Ghantous, Brice Minaud, Ky Nguyen (Nguyễn Ngọc Kỷ), Paola de Perthuis, Guillaume Renaut, Mélissa Rossi, Éric Sageloli, Robert Schädlich et Quoc-Huy Vu.\par

Je remercie aussi, pour les mêmes raisons, mes collègues de Thales à Gennevilliers, que furent, entre autres, Zoé Amblard, Loïc Demange, Éric Garrido, Matthieu Giraud, David Lefranc, Lucie Mousson et Philippe Painchault.

Pour la « préservation de l’anonymat » de mes proches, je ne vais pas les citer ici mais je pense bien sûr à eux avec gratitude car, dans cette aventure qu’a été ma formation doctorale, il est arrivé qu’en plus de m’offrir leur soutien, ils m’aient offert leur secours.\par

À présent, cher lecteur, si vous vous trouvez au nombre de ceux à qui j’ai coupablement omis de témoigner de ma reconnaissance, je n’aurai qu’un mot à dire pour ma défense : merci !\par


%et aux docteurs qui en sont fraîchement émoulu
%Ainsi, tant po

%de sorte que j’ai pleinement pu profiter tant de son aide sur des questions d’ordre pratique que de ses conseils scientifiques éclairées.\par

% m’a toujours assuré de sa disponibilité au moindre besoin que j’avais de profiter de son aide pratique ou de ses conseils scientifiques éclairés.
% pour répondre au moindre de mes besoins 
%Éric Garrido,
% Philippe
% Simon Abelard
% Renaud Dubois

%à la moindre nécessité de
% m’apporter toute l’aide et tous les éclairages dont 
% et qui m’a guidé de main de maître (de conférence) à travers ces années.

%Je pense à mes proches, que je m’abstiens de citer ici mais qui m’ont offert leur soutien.

%les pieds en éventail sur l’île paradisiaque du pacifique que leur momumentale/mirifique traitement de chercheur leur ont sans doute permis d’acheter. J’ai apprécié la pertinence des commentaires qu’il ont faits au sujet du manuscrit.\par

% je n’ai pas prévu une 
%Au moment d’écrire ses lignes, j’ai une pensée pour les membres du laboratoire de Thales à Palaiseau dans lequel j’ai fait mon stage. Merci à Olivier Bettan qui a su se rappeler à mon bon souvenir.\par

%se sont attelé à lire mon manuscrit au milieu de l’été et à rédiger un rapport au beau milieu de l’été en un temps restraint.



%Pour sa gentillesse à toute épreuve, sa disponibilité sans faille, et sa guidance de qualité, j’adresse à Céline Chevalier mes plus chaleureux remerciements.



%Pour sa disponibilité à toute épreuve
%parce qu’elle m’a toujours assuré de sa dispoini

%Pour l’assurance qu’elle m’a toujours apportée de 

%Pour sa gentillesse à toute épreuve, pour l’empressement qu’elle a témoigné à me faire savoir qu’elle pouvait se rendre disponible à tout moment, pour 

%qui m’a toujours assuré de sa disponibilité,

%Céline a toujours tenu à me faire savoir qu’elle était disponible en cas de besoin.

%Pour cela, je lui adresse mes plus chaleureux remerciements.




%Pour l’empressement qu’elle a toujours eu à me savoir qu’elle serait disponible

%Pour les efforts qu’elle a consenti pour se rendre disponible, pour sa gentillesse sans faille, j’adresse à Céline mes plus chaleureux remerciements.




%À Thomas Ricosset, qui fut mon second directeur de thèse pour Thales, je tiens à exprimer ma gratitude.
%Ses conseils avisés/pertinents ont participé à rendre l’expérience de préparation de cette thèse des plus formatrices.

%Gentillesse.
%Assistance.
%Prévenance.
%Au grand manitou, David Pointcheval, chef de laboratoire de cryptographie et du département d’informatique de l’École normale supérieure, je 
%À Éric Sageloli, dont l’efficacité est impressionnante.
%Éric Garrido.
%Grâce à eux, cette expérience s’est averé très formatrice.

%malgré des conditions difficiles, des restrictions qui nous ont obligé à une éloignement.


%Céline Chevalier
%Thomas Ricosset
%David Pointcheval
%Éric Sageloli
%Les membres du labo de l’ENS
%Les membres de l’équipe de Thales
%Les rapporteurs et les membres du jury.


%À ... je veux dire dire toute ma gratitude.
%À ... je tiens à témoigner de.

%Qui ont su répondre à mes questions et parfois même les anticiper.

%Je pense à mes proches, que la discrétion m’impose de ne pas citer ici mais qui m’ont offert leur soutien en des moments, à des moments.


%\lipsum[1-8]

% chapter acknowledgements (end)
