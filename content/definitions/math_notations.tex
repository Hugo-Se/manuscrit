% !TEX root = definitions_master.tex

\section{Mathematical Notations} % (fold)
\label{sec:math_notations}

\paragraph{Sets and rings.} % (fold)
\label{par:def_sets_rings}

The set of integers is denoted $\ZZ$, the set of non-negative integers $\NN$.
If $a$ and $b$ are two integers such that $a \leq b$, we denote by $\bb{a,b}$ the set $\{x \in \ZZ | a \leq x \leq b\}$ of integers between $a$ and $b$ (both included).
Also, $\ZZ_n$ is the ring $\ZZ/n \ZZ$ of integers modulo the integer $n$.
For a prime integer $p$, $\FF_p = \ZZ_p$ is the field with $p$ elements.
$\varphi(\cdot)$ is the Euler totient function.
For any set $S$, $\cP(S)$ is the set of all finite subsets of $S$.
% paragraph def_sets_rings (end)

\paragraph{Bilinear Groups} % (fold)
\label{par:def_bilinear_groups}

Some of the cryptographic primitives we will use an additional structure on groups called a pairing (a.k.a. bilinear groups).
Let $\GG_1$, $\GG_2$ and $\GG_T$ be three cyclic groups of the same order $N$ and with respective generator $g_1$, $g_2$ and $g_T$.
$(\GG_1, \GG_2, \GG_T, e)$ is called a \emph{bilinear group} if $e : \GG_1 \times \GG_2 \rightarrow \GG_T$ satisfies the following properties:
\begin{itemize}
	\item For all $(a,b) \in \ZZ^2_N$, $e(g_1^a, g_2^b) = e(g_1,g_2)^{a b}$ (bilinearity);
	\item The element $e(g_1,g_2)$ generates $\GG_T$ (non-degeneracy);
	\item $e(\cdot,\cdot)$ is efficiently computable.
\end{itemize}
We explain the next paragraphs what efficient means.
Without loss of generality, we can suppose that $e(g_1,g_2) = g_T$.
We call bilinear groups with $\GG_1 = \GG_2$ Type-1 (or symmetric) bilinear groups, and in this case, we suppose that $g = g_1 = g_2$.
If $\GG_1 \neq \GG_2$, and that there is no efficiently computable isomorphism between $\GG_1$ and $\GG_2$, it is a Type-3 (asymmetric) bilinear group.
% subsection def_bilinear_groups (end)

\paragraph{Bit strings.} % (fold)
\label{par:def_bit_strings}


A bit $b$ is an element in $\zo$; a bit string $s$ of length $n$ is a vector of $n$ bits.
$\varepsilon$ denotes the \emph{empty} bit string of size 0. 
The set of bit strings of size $n$ is hence denoted $\zo^n$, and the set of bit strings of finite length is $\zo^* = \cup_{n \geq 0} \zo^n$.
The concatenation of two bit strings $x$ and $y$ is denoted $x||y$.

% paragraph def_bit_strings (end)


\paragraph{Distributions and Probabilities} % (fold)
\label{par:def_distrib_probas}

For a finite set $S$, $s \getsr S$ means that $s$ is sampled uniformly at random from $S$.
The probability of an event $E$ to occur is denoted $\Prob[E]$.
Let $X$ be a random variable. The expected value of $X$ is $\Exp[X]$, and $\Var[X]$ its variance.
The entropy and min-entropy of a discrete random variable $X$ taking value $\{x_1, \dots, x_n\}$ are denoted $H_1(X)$ and $H_{\infty}(X)$ and are defined as
\begin{align*}
	\Entropy(X) &= - \sum_{i=1}^{n} \Prob[X = x_i] \cdot \log \Prob[X = x_i], \\
	\text{and } \MinEntropy(X) &= - \min_{1 \leq i \leq n} \{ \log \Prob[X = x_i] \} = \max_{1 \leq i \leq n} \{ - \log \Prob[X = x_i] \}
\end{align*}
where $\log$ is the base-2 logarithm.
For two random discrete random variable $Y$ and $Z$ over the set $\{x_1, \dots, x_n\}$, the  Kullback–Leibler divergence from $Y$ to $Z$, $\Div(Y || Z)$, is defined as
\[
	\Div(Y || Z) = \sum_{i=1}^{n} \Prob[Y = x_i] \log \frac{\Prob[Y = x_i]}{\Prob[Z = x_i]}.
\]

Finally, we define the distance $\Delta(\cD_1, \cD_2)$ between the two distributions $\cD_1$ and $\cD_2$ over a set $X$ as
\[
	\Delta(\cD_1, \cD_2) = \max_{x \in X} \left|\Prob[Y_1 = x] -  \Prob[Y_2 = x]\right|
\]
where $Y_1$ (resp $Y_2$) is a random variable following the distribution $\cD_1$ (resp. $\cD_2$).
% paragraph def_distrib_probas (end)


\paragraph{Asymptotics.} % (fold)
\label{par:def_asymptotics}

For asymptotics, we use the standard Landau notations $\bigO{\cdot}$, $\smallO{\cdot}$, $\bigOmega{.}$, $\smallOmega{.}$ and $\bigsmallO{.}$.
We also use $\bigOtilde{.}$ to hide poly-logarithmic factors in asymptotics:
\[
	f(n) = \bigOtilde{g(n)} \Leftrightarrow \exists c \in \NN \text{ such that } f(n) = \bigO{g(n) \log^c(n)}.
\]
In the following, $\poly[n]$ denotes an unspecified function $f(n) = \bigO{n^c}$ for some fixed constant $c$, and $\negl[n]$ is a negligible function $f$ such that $f(n) = \smallO{n^{-c}}$ for any constant $c > 0$. 

% paragraph def_asymptotics (end)


\paragraph{Algorithms, Turing machines, and Oracles} % (fold)
\label{par:def_tm_oracles}

Algorithms are programs for Turing machines.
They can be probabilistic, and use a tape of the Turing machine filled with random bits (also called random coins).
By default algorithms are probabilistic. 
For an algorithm $A$, $y \gets A(x)$ means that $A$ is run on input $x$ (with fresh random coins if $A$ is probabilistic), and that the result is stored in $y$.
More generally $y \gets a$ states that the result of the evaluation of the expression $a$ is stored in the variable $y$.

An interactive Turing machine is a special kind of Turing machines able to communicate with external algorithms. 
To do so, the interactive Turing machine uses additional tapes to communicate with the other Turing machines, namely an input tape to send messages, and an output tape to receive messages.
For an interactive Turing machine $\cT$, the Turing machines $\cT$ has access to are called its \emph{oracles}.
We write $\cT^{O_1, O_2}(x)$ to say that the Turing machine $\cT$ is called with input $x$ and has access to the oracles $O_1$ and $O_2$.


For a distribution $\cD$, $A(\cD)$ denotes the output of the execution of $A$ on an input $x$ sampled from the distribution $\cD$.


In this manuscript, we will often abuse notations, and identify a Turing machine and the algorithm it runs. 

Finally, we will say that an algorithm is \emph{efficient} if it runs in time polynomial in the size of its arguments.

% paragraph def_tm_oracles (end)

\paragraph{Protocols.} % (fold)
\label{par:def_protocols}
	A two-party protocol $P = (A_1, A_2)$ is a pair of algorithms $A_1$ and $A_2$, interactively executed by a pair of two Turing machines $\cT_1$ and $\cT_2$. 
	% In the paper, we will construct and use some two-party protocols, involving a client $C$ and a server $S$.
	We will denote the execution of the protocol $P$ as 
	\[
		P(\mathtt{input}_1; \mathtt{input}_2) = (A_1(\mathtt{input}_1), A_2(\mathtt{input}_2)),
	\]
	meaning that $A_1$ (resp. $A_2$) is executed by $\cT_1$ (resp. $\cT_2$) with input $\mathtt{input}_1$ (resp. $\mathtt{input}_2$).
	We write
	\[
		(\mathtt{out}_1; \mathtt{out}_2) \getsr A_1(\mathtt{input}_1) \leftrightarrow A_2(\mathtt{input}_2)
	\]
	to mean that $\mathtt{out}_1$ and $\mathtt{out}_2$ are the outputs of the interaction between $A_1$ on input $\mathtt{input}_1$ and $A_2$ on input $\mathtt{input}_2$, respectively.
	We also simplify this notation and denote the result of the execution of $P$ as
	\[
		(\mathtt{out}_1; \mathtt{out}_2) \getsr  P(\mathtt{input}_1; \mathtt{input}_2)
	\]
	In this formalism, we consider the messages $\tau_{1 \rightarrow 2}$ (resp. $\tau_{1 \leftarrow 2}$)  sent by $\cT_1$ to $\cT_2$ (resp. $\cT_2$ to $\cT_1$) as part of the output $\mathtt{out}_1$ (resp. $\mathtt{out}_2$).
	These messages are called the \emph{transcript} of $\cT_1$ (resp. $\cT_2$). 
	Transcripts might be omitted from the output of the protocol for simplicity of the notations.
% paragraph def_protocols (end)


\paragraph{Miscellaneous.} % (fold)
\label{par:def_miscellaneous}
As mentioned earlier, the base-2 logarithm of the value $x$ is $\log x$. 
When the variable $T$ is a dictionary, $T[v]$ denotes the item associated to $v$, if there is one, whereas $\bot$ denotes the absence of this item.

% paragraph def_miscellaneous (end)

% section math_notations (end)