\usepackage{lmodern}
\usepackage{lipsum}
\usepackage{couverture/modèle_PSL}

%\title{Protocoles cryptographiques pour un anonymat post-quantique}
%\title{Protocoles cryptographiques post-quantiques pour la garantie de l’anonymat et du secret des messages}
%\title{Protocoles cryptographiques post-quantiques :\newline étude et mise en œuvre}
%\title{Protocoles cryptographiques post-quantiques pour la garantie du secret des communications et des identités}
\title{\textbf{Protocoles cryptographiques post-quantiques\\de préservation de l’anonymat et du secret\\ des communications}}
%\title{\textbf{Étude et mise en œuvre de protocoles cryptographiques post-quantiques au service de l’anonymat et du secret des communications}}
% anonymat et secret des communications
% conception (mise en œuvre) et analyse de protocoles cryptographiques post-quantiques
% Mise en œuvre et étude de protocoles cryptographiques post-quantiques.
% protocole cryptographique 
% l’anonymat et la confidentialité
% Anonymat et confidentialité
% ÉTUDE ET MISE EN ŒUVRE

\author{Hugo SENET}

\institute{l’École normale supérieure\newline et à Thales}
%\doctoralschool{Sciences\hskip.2cm mathématiques de Paris-Centre}{386}
\doctoralschool{Sciences mathématiques\newline de Paris-Centre}{386}
\specialty{Informatique}
\date{18 septembre 2023}

% cotutelle
% \entitle{Thesis Subject in English}
%\otherinstitute{CEA Saclay}
%\logootherinstitute{logotype_Thales.png}

\jurymember{1}{Dương Hiệu PHAN}{Télécom Paris}{Rapporteur} % Professeur
\jurymember{2}{Adeline ROUX-LANGLOIS}{CNRS}{Rapporteur} % Chargée de recherche, Caen
\jurymember{3}{David POINTCHEVAL}{École normale supérieure, CNRS}{Examinateur} % Directeur de recherche
\jurymember{4}{Olivier BLAZY}{École polytechnique}{Examinateur} % Professeur
\jurymember{5}{Damien VERGNAUD}{Sorbonne Université}{Examinateur}
\jurymember{6}{Céline CHEVALIER}{École normale supérieure\newline Université Paris-Panthéon-Assas\newline}{Directrice de thèse}
\jurymember{7}{Thomas RICOSSET}{\vspace{-0.65cm}Thales}{Directeur de thèse \emph{en entreprise}}


%\frabstract{Pour parer à la menace que représenterait l’essor de calculateurs quantiques puissants sur nombre de systèmes cryptographiques [ayant cours].}
\frabstract{
Pour parer à la menace que représenterait l’essor des calculateurs quantiques sur nombre de systèmes cryptographiques ayant cours actuellement, d’importants efforts sont fait pour que voient le jour et se développent des systèmes dits « post-quantiques » de chiffrement, d’authentification et d’identification, analogues aux systèmes classiques mais qui résisteraient à des attaques rendues possibles par de tels calculateurs, devenus suffisamment puissants.

%Les accréditations anonymes, qui permettent de vérifier les droits d’un individu à accéder à un service

%Les systèmes d’accréditations anonymes permettent la préservation du secret de l’identité à tout utilisateur.

Dans cet esprit, mes collègues et moi-même nous sommes penchés sur les systèmes d’accréditations anonymes, qui permettent à un fournisseur de services de vérifier les droits d’un utilisateur à certains de ces services en vertu de la garantie, donnée antérieurement par une autorité, du fait que cet utilisateur possède bien les attributs requis pour en bénéficier, tout en préservant le secret des autres attributs, comme par exemple l’identité de l’utilisateur, dont l’autorité aurait eu à prendre connaissance.
Nous nous sommes employés à étudier certains des systèmes d’accréditations anonymes les plus avancés décrits dans la littérature, à partir desquels nous avons construit et mise en œuvre au niveau informatique notre système, en apportant à nos programmes diverses optimisations en temps et en mémoire.

Le protocole Signal, sur lequel se fondent nombre de systèmes de messagerie instantanée actuels, est lui aussi appelé à connaître une adaptation post-quantique. Dans un article, nous en proposons une preuve formelle réalisée avec Tamarin avec un accent mis sur les deux principales composantes, le protocole d’échange de clefs X3DH et le protocole du double cliquet.

%, que nous avons ensuite mise en œuvre de façon fine sur le plan informatique, avec des optimisations de nos programmes en mémoire et en temps.
%à apporter notre pierre à l’édifice

}
\frkeywords{Cryptographie post-quantique, accréditations anonymes, cryptographie asymétrique, chiffrement vérifiable, preuves à divulgation nulle de connaissance, réseaux euclidiens.}
