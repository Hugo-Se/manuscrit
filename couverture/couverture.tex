\usepackage{lmodern}
\usepackage{lipsum}
\usepackage{couverture/modèle_PSL}

%\title{Protocoles cryptographiques pour un anonymat post-quantique}
%\title{Protocoles cryptographiques post-quantiques pour la garantie de l’anonymat et du secret des messages}
%\title{Protocoles cryptographiques post-quantiques :\newline étude et mise en œuvre}
\title{Protocoles cryptographiques post-quantiques pour la garantie du secret des communications et des identités}

\author{Hugo SENET}

\institute{l’École normale supérieure\newline et à Thales}
%\doctoralschool{Sciences\hskip.2cm mathématiques de Paris-Centre}{386}
\doctoralschool{Sciences mathématiques\newline de Paris-Centre}{386}
\specialty{Informatique}
\date{18 septembre 2023}

% cotutelle
% \entitle{Thesis Subject in English}
%\otherinstitute{CEA Saclay}
%\logootherinstitute{logotype_Thales.png}

\jurymember{1}{Dương Hiệu PHAN}{Telecom Paris}{Rapporteur} % Professeur
\jurymember{2}{Adeline ROUX-LANGLOIS}{CNRS}{Rapporteur} % Chargée de recherche, Caen
\jurymember{3}{David POINTCHEVAL}{École normale supérieure, CNRS}{Examinateur} % Directeur de recherche
\jurymember{4}{Olivier BLAZY}{École polytechnique}{Examinateur} % Professeur
\jurymember{5}{Damien VERGNAUD}{Sorbonne Université}{Examinateur}
\jurymember{6}{Céline CHEVALIER}{École normale supérieure\newline Université Paris-Panthéon-Assas\newline}{Directrice de thèse}
\jurymember{7}{Thomas RICOSSET}{\vspace{-0.65cm}Thales}{Directeur de thèse \emph{en entreprise}}


%\frabstract{Pour parer à la menace que représenterait l’essor de calculateurs quantiques puissants sur nombre de systèmes cryptographiques [ayant cours].}
\frabstract{
Pour parer à la menace que représenterait l’essor des calculateurs quantiques sur nombre de systèmes cryptographiques ayant cours actuellement, d’importants efforts sont fait pour que voient le jour et se développent des systèmes, dits « post-quantiques », de chiffrement, d’authentification et d’identification qui résisteraient à des attaques qui s’appuieraient sur de tels calculateurs, devenus suffisamment puissants.
}
\frkeywords{réseaux euclidiens, cryptographie post-quantique}
