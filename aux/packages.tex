% !TEX root = ../master.tex

% Use utf-8 encoding for foreign characters
%\usepackage[utf8]{inputenc} %%%%%%%%%%
%\usepackage[russian,french,english]{babel} %%%%%%%%%%


\usepackage{fontspec} %%%%%%%%%%
\usepackage{polyglossia} %%%%%%%%%%
\usepackage{realscripts} %%%%%%%%%%
\usepackage{newunicodechar} %%%



\usepackage{pdfpages} % For front and back covers, warning, when scehack is used, must be loaded before mathtools

\usepackage{amsmath}
\usepackage{amsfonts}
\usepackage{amssymb}

\usepackage{mathtools}
\usepackage{mathdots}
\usepackage{stmaryrd}

\usepackage{enumitem, framed}



%% For gorgeous subfigures
\usepackage{caption}
\captionsetup{justification=justified,labelfont=bf,labelsep=endash}
\usepackage{subcaption}
\usepackage{dblfloatfix}                 


\usepackage{multicol}
\usepackage{boxedminipage}

%----------------------------------------------------------------------
%                     Floats and Tables
%----------------------------------------------------------------------

\usepackage{booktabs}

%% For footnotes in tables
\usepackage{footnote}
\makesavenoteenv{tabular}
\makesavenoteenv{table}
\makesavenoteenv{table*}
\usepackage{multirow}

\usepackage{rotating}

%----------------------------------------------------------------------
%                     Minitoc
%----------------------------------------------------------------------

\onlyif{chap_minitoc}{
\usepackage[english]{minitoc}
\setcounter{minitocdepth}{2}
}

%----------------------------------------------------------------------
%                     Todonotes
%----------------------------------------------------------------------

\iftoggle{allowtodo}{
  \iftoggle{showtodo}{
  	\iftoggle{inlinetodo}{
		\newcommand{\todo}[1]{\textcolor{red}{[{\footnotesize {\bf TODO:} { {#1}}}]}}
		
	}{
		\usepackage[textsize=small]{todonotes}
	}
  }{
    \usepackage[disable]{todonotes}
	\renewcommand{\todo}[1]{}
  }
  % add xspace to todo command (http://tex.stackexchange.com/a/68741)
  \usepackage{xspace}
  % \makeatletter
  \expandafter\apptocmd\csname\string\todo\endcsname{\xspace}{}{}
  % \makeatother
}{
}

%----------------------------------------------------------------------
%                     Algorithms
%----------------------------------------------------------------------

\usepackage{algorithm}
\usepackage{algorithmicx}
\usepackage[]{algpseudocode}
\newcommand{\algsep}{\begin{center}\rule{\textwidth}{0.4pt}\end{center}}

\renewcommand{\thealgorithm}{\arabic{chapter}.\arabic{algorithm}}


% !TEX root = ../master.tex


% New 'keywords' for algorithmicx
\newcommand{\InputA}[1]{\State \textbf{Input A:} #1}
\newcommand{\InputB}[1]{\State \textbf{Input B:} #1}
\newcommand{\OutputA}[1]{\State \textbf{Output A:} #1}
\newcommand{\OutputB}[1]{\State \textbf{Output B:} #1}
\algnewcommand\algorithmicto{\textbf{to}}
\algrenewtext{For}[3]%
  {\algorithmicfor\ $#1 = #2$ \algorithmicto\ $#3$ \algorithmicdo}

\algblockdefx[IFRED]{IfRed}{EndIfRed}
	[1]{\red{\algorithmicif\ {#1}}}
   	[0]{\red{\algorithmicend\ \algorithmicif}}

\algcblockx{IFRED}{ElseRed}{EndIfRed}
	[0]{\red{\algorithmicelse\ }}
	[0]{\red{\algorithmicend\ \algorithmicif}}
   
\algblockdefx[FORALLRED]{ForAllRed}{EndForRed}
   [1]{\red{\algorithmicforall\ {#1}}}
   [0]{\red{\algorithmicend\ \algorithmicfor}}

\algloopdefx[SMALLIF]{SmallIf}
	[1]{\algorithmicif\ {#1}}   
   
\algloopdefx[SMALLIFRED]{SmallIfRed}
	[1]{\red{\algorithmicif\ {#1}}}  
	 
\algblockdefx[GAME]{GameBlock}{EndGameBlock}
   [1]{{#1} Game:}
   [0]{}
\algtext*{EndGameBlock} % remove the EndGame line

\newcommand{\IfLine}[1]{\State \algorithmicif\ {#1}}   
\newcommand{\ElseLine}{\State \algorithmicelse\ }   
\newcommand{\IfLineRed}[1]{\State \red{\algorithmicif\ {#1}}}   
  
\algdef{SE}[DOWHILE]{Do}{doWhile}{\algorithmicdo}[1]{\algorithmicwhile\ #1}%

\algrenewcommand\algorithmicindent{1.0em}%
\algrenewcommand\alglinenumber[1]{\scriptsize #1:}
\algrenewcommand{\algorithmiccomment}[1]{\hfill\hspace{.2cm} $\triangleright$\ \textit{#1}}

\makeatletter
\newcommand{\setalglineno}[1]{%
  \setcounter{ALG@line}{\numexpr#1-1}}
\makeatother



% to create nice, two columns, algorithms
\newcommand{\twocolsalgsrightmargin}{.7em}
\newcommand{\twocolsalgs}[2]{%
	\begin{varwidth}[t]{0.48\textwidth}%
		{#1}%
	\end{varwidth}%
\hfill%
	\begin{varwidth}[t]{0.48\textwidth}%
		{#2}%
	\end{varwidth}%
\hspace{\twocolsalgsrightmargin}%
}


\usepackage{float}

\newfloat{protocol}{tbph}{lop}
\floatname{protocol}{Protocol}


%----------------------------------------------------------------------
%                     Theorems
%----------------------------------------------------------------------

\usepackage{amsthm}

% To repeat theorems numbers easily (e.g. when deferring the proof)
\makeatletter
\newtheorem*{rep@theorem}{\rep@title}
\newcommand{\newreptheorem}[2]{%
\newenvironment{rep#1}[1]{%
 \def\rep@title{#2 \ref{##1}}%
 \begin{rep@theorem}}%
 {\end{rep@theorem}}}
\makeatother

\newtheorem{theorem}{Theorem}[chapter]
\newreptheorem{theorem}{Theorem}

\newtheorem{proposition}[theorem]{Proposition}
% \newtheorem{informaltheorem}[theorem]{Theorem}
% \newtheorem{informalcorollary}[theorem]{Corollary}
% \newtheorem{informalclaim}[theorem]{Claim}
\newtheorem{definition}{Definition}[chapter]
\newtheorem{claim}[theorem]{Claim}
% \newtheorem{remark}{Remark}[section]
% \newtheorem{remark}[theorem]{Remark}
\newtheorem{lemma}[theorem]{Lemma}
\newtheorem{corollary}[theorem]{Corollary}
% \newtheorem{fact}[theorem]{Fact}
% \newtheorem{assumption}{Assumption}


%----------------------------------------------------------------------
%                     Crypto
%----------------------------------------------------------------------

\usepackage[lambda,advantage,adversary,landau,sets,notions,logic,ff,primitives,events,asymptotics,keys]{cryptocode}

% Advandtage style
\renewcommand{\pcadvantagesuperstyle }[1]{\mathrm{#1}}
% \renewcommand{\pcadvantagesubstyle }[1]{#1}

% Adversary style
% \renewcommand{\pcadvstyle }[1]{\ mathcal{#1}}

%----------------------------------------------------------------------
%                     TikZ
%----------------------------------------------------------------------


\usepackage{tikz}
\usetikzlibrary{shapes,positioning,calc}
\usetikzlibrary{arrows}
\usetikzlibrary{fit}

\def\ruleoffset{0.5em}


%----------------------------------------------------------------------
%                     GnuPlot (with TikZ)
%----------------------------------------------------------------------
\usepackage{gnuplot_sty/gnuplot-lua-tikz}

%----------------------------------------------------------------------
%                     Import (useful when using a lot of files)
%----------------------------------------------------------------------
\usepackage{import}

%----------------------------------------------------------------------
%                     Figures and Units
%----------------------------------------------------------------------
\usepackage[binary-units,mode=text]{siunitx}

%----------------------------------------------------------------------
%                     Hyperref at last
%----------------------------------------------------------------------

\definecolor{blue_link}{RGB}{0,0,150}

\usepackage[pageanchor=true,
			plainpages=false,
			pdfpagelabels=true,
			bookmarksopen=true,
			bookmarksopenlevel=1
			]{hyperref}


\setdefaultlanguage{french} %%%%%%%%%%
\setotherlanguage{english} %%%%%%%%%%

\hypersetup{
  % linktoc        = page,
  % pdfpagemode    = UseOutlines,
  colorlinks     = \iftoggle{printversion}{false}{true},
  linkcolor      = blue_link,
  citecolor      = blue_link,
  urlcolor       = blue_link,
  bookmarksdepth = 3
}

\makeatletter
\AtBeginDocument{
  \hypersetup{
    pdftitle  = {\@title},
    pdfauthor = {\@author}
  }
}
\makeatother

\renewcommand*{\theHsection}{ch.\the\value{chapter}.sec.\the\value{section}}

\renewcommand\UrlFont{\small\ttfamily}
